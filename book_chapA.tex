\chapter{De ASCII-tabel}
\label{cha:asciitabel}
\thispagestyle{empty}

We hebben gezien dat binaire coderingen worden gebruikt om
numerieke gegevens (of informatie) weer te geven. Maar niet alle gegevens zijn numeriek.
Informatie kan ook bestaan uit letters en leestekens. Om er voor te zorgen dat
computers informatie kunnen uitwisselen is de ASCII-code bedacht.

\begin{table}[!p]
\caption{De ASCII-code~\cite{eijkhout2010ascii}.}
\label{tab:talascii-code}

\resizebox{\textwidth}{!}{%%%%%%%%%%%%%%%%%%%%%%%%%%%%%%%%%%%%%%%%%%%%%%%%%%%%%%%%%%%%%%%%%%%%%%%%
%%                                                                    %%
%%                     A S C I I    wall chart                        %%
%%                                                                    %%
%% by Victor Eijkhout                                                 %%
%% victor@eijkhout.net                                                %%
%%                                                                    %%
%%%%%%%%%%%%%%%%%%%%%%%%%%%%%%%%%%%%%%%%%%%%%%%%%%%%%%%%%%%%%%%%%%%%%%%%
%% Copyright 2009 Victor Eijkhout
%
% This work may be distributed and/or modified under the
% conditions of the LaTeX Project Public License, either version 1.3
% of this license or (at your option) any later version.
% The latest version of this license is in
%   http://www.latex-project.org/lppl.txt
% and version 1.3 or later is part of all distributions of LaTeX
% version 2005/12/01 or later.
%
% This work has the LPPL maintenance status `maintained'.
% 
% The Current Maintainer of this work is Victor Eijkhout.
%
% This work consists of this file.
%
%% Choose your favourite format:
%
%\nopagenumbers %% 2 lines
%\vsize=28cm    %% for PLAIN TeX
%\documentstyle{article}            %% 4 lines for LaTeX
%\begin{document}                   %% not that it matters anything,
%\pagestyle{empty}                  %% rest of the document
%\setlength{\textheight}{28cm}      %% is 'pure' TeX.
%%%%%% and don't forget the \bye / \end{document} at the end!! %%%%%%%
%% fonts
\font\bitfont=cmr7 \fontdimen3\bitfont=3mm
\font\codefont=cmr5
\font\namefont=cmss10 scaled 1200
\font\titlefont=cmss10 scaled 1440
\font\commentfont=cmss10
%% counts and dimens
\newdimen\thinlinewidth \thinlinewidth=.25mm
\newdimen\fatlinewidth \fatlinewidth=.5mm
\newdimen\rowheight \rowheight=1cm
\newdimen\colwidth  \colwidth=1.6cm
\newdimen\Colwidth \Colwidth=2\colwidth
  \advance\Colwidth by \thinlinewidth
\newdimen\topwhite \topwhite=2pt
\newdimen\botwhite \botwhite=3pt
\newdimen\leftwhite \leftwhite=2pt
\newdimen\rightwhite \rightwhite=2pt
\newcount\rowcount \rowcount=-1 %% note!
\newcount\colcount \colcount=0
\newcount\thenumber
%% tidbits
\def\\{$\backslash$}
\def\thinline{\vrule width \thinlinewidth}
\def\fatline{\vrule width \fatlinewidth}
\tolerance=10000
\vbadness=10000
%% code conversion
\def\calcnumber{{\multiply\colcount by 16
                 \advance\colcount by \rowcount
                 \global\thenumber=\colcount}}
\def\deccode{\number\thenumber}
\def\octcode{{\ifnum\thenumber>63
                            \advance\thenumber by -64
                            \count0=\thenumber \divide\count0 by 8
                            1\number\count0
              \else         \count0=\thenumber \divide\count0 by 8
                            \ifnum\count0>0 \number\count0 \fi\fi
              \multiply\count0 by 8
              \advance\thenumber by -\count0
              \number\thenumber}}
\def\hexdigit#1{\ifcase#1 0\or 1\or 2\or 3\or 4\or 5\or 6\or 7\or
                          8\or 9\or A\or B\or C\or D\or E\or F\or
                          \edef\tmp{\message{illegal hex digit
                                        \number#1}}\tmp
                          \fi}
\def\hexcode{{\count0=\thenumber \divide\count0 by 16
              \ifnum\count0>0 \hexdigit{\count0}\fi
              \multiply\count0 by 16
              \advance\thenumber by -\count0 \count0=\thenumber
              \hexdigit{\count0}}}
%% the heading
%% Jesse 10-10-2016: Changed 1.2 to 1.3
\def\threebit#1#2#3{\vbox to 1.3\rowheight{\bitfont
                      \vskip\topwhite
                      \hbox to \colwidth{\hskip\leftwhite#1\hfil}
                      \vfil
                      \hbox to \colwidth{\hfil#2\hfil}
                      \vfil
                      \hbox to \colwidth{\hfil#3\hskip\rightwhite}
                      \vskip\botwhite}}
\def\comment#1{\vbox to \colwidth{\hrule height 0mm depth .25mm
                      \vfil
                      \hbox to \Colwidth{\commentfont\hfil#1\hfil}
                      \vfil}}
\def\dcomment#1#2{\vbox to \colwidth{\hrule height 0mm depth .25mm
                      \vfil
                      \hbox to \Colwidth{\commentfont\hfil#1\hfil}
                      \vskip \botwhite
                      \hbox to \Colwidth{\commentfont\hfil#2\hfil}
                      \vfil}}
\def\bithead{\vbox to \colwidth{\hsize=1.5\colwidth
                   \vskip\topwhite
                   \hbox to \hsize{\commentfont\hfil BITS\hfil}
                   \vfil
                   \hbox to \hsize{\bitfont\ b4 b3 b2 b1 }
                   \vskip\botwhite}}
%% routines for single chars
\def\fourbit#1\fb{\vbox to \rowheight{
                     \vfil
                     \hbox to 1.5\colwidth{\bitfont #1\ }
                     \vfil}%
                  \global\advance\rowcount by 1
                  \global\colcount=0}
\def\asc#1\ii{\calcnumber
              \vbox to \rowheight{\offinterlineskip
                     \vskip\topwhite
                     \hbox to \colwidth{\codefont
                                        \hskip\leftwhite
                                        \deccode\hfil}
                     \vfil
                     \hbox to \colwidth{\vrule width 0cm
                                              height 10pt depth 2pt
                                        \namefont
                                        \hfil#1\hfil}
                     \vfil
                     \hbox to \colwidth{\codefont
                                        \hskip\leftwhite
                                        \hexcode\hfil\octcode
                                        \hskip\rightwhite}
                     \vskip\botwhite}%
              \global\advance\colcount by 1}
%%%%%%%%%%%%%%%%% and now the table itself %%%%%%%%%%%%%%%%%%%%%%%%%
% Jesse 10-10-2016: added hskip-1.5cm brute force
\hskip-1.5cm\vbox{
\halign{\fourbit#\fb&\fatline\asc#\ii&\thinline\asc#\ii&
                     \fatline\asc#\ii&\thinline\asc#\ii&
                     \fatline\asc#\ii&\thinline\asc#\ii&
                     \fatline\asc#\ii&\thinline\asc#\ii\fatline\cr
          % Jesse 10-10-2016: Double \omit to center title better
          \omit&\omit&\multispan7 \hskip\thinlinewidth
                            \titlefont ASCII CONTROL CODE CHART\hfil\cr
    \noalign{\vskip3mm \hrule}
          \omit\hfil\threebit{b7}{b6}{b5}
                 &\omit\fatline\threebit000&\omit\thinline\threebit001%
                 &\omit\fatline\threebit010&\omit\thinline\threebit011%
                 &\omit\fatline\threebit100&\omit\thinline\threebit101%
                 &\omit\fatline\threebit110&\omit\thinline\threebit111%
                                                 \fatline\cr
    \noalign{\vskip-.5mm} %brute force
          \omit\bithead
                 &\omit\fatline\comment{CONTROL}\span\omit
                 &\omit\fatline\dcomment{SYMBOLS}{NUMBERS}\span\omit
                 &\omit\fatline\comment{UPPER CASE}\span\omit
                 &\omit\fatline\comment{LOWER CASE}\span\omit\hfil\fatline\cr
    \noalign{\hrule}
      {} 0 0 0 0&NUL&DLE&SP     &0  &@  &P       &` &p  \cr\noalign{\hrule}
      {} 0 0 0 1&SOH&DC1&!      &1  &A  &Q       &a &q  \cr\noalign{\hrule}
      {} 0 0 1 0&STX&DC2&"      &2  &B  &R       &b &r  \cr\noalign{\hrule}
      {} 0 0 1 1&ETX&DC3&\#     &3  &C  &S       &c &s  \cr\noalign{\hrule}
      {} 0 1 0 0&EOT&DC4&\$     &4  &D  &T       &d &t  \cr\noalign{\hrule}
      {} 0 1 0 1&ENQ&NAK&\%     &5  &E  &U       &e &u  \cr\noalign{\hrule}
      {} 0 1 1 0&ACK&SYN&\&     &6  &F  &V       &f &v  \cr\noalign{\hrule}
      {} 0 1 1 1&BEL&ETB&'      &7  &G  &W       &g &w  \cr\noalign{\hrule}
      {} 1 0 0 0&BS &CAN&(      &8  &H  &X       &h &x  \cr\noalign{\hrule}
      {} 1 0 0 1&HT &EM &)      &9  &I  &Y       &i &y  \cr\noalign{\hrule}
      {} 1 0 1 0&LF &SUB&*      &:  &J  &Z       &j &z  \cr\noalign{\hrule}
      {} 1 0 1 1&VT &ESC&+      &;  &K  &[       &k &$\{$\cr  \noalign{\hrule}
      {} 1 1 0 0&FF &FS &,      &$<$&L  &\\      &l &$|$ \cr  \noalign{\hrule}
      {} 1 1 0 1&CR &GS &$-$    &=  &M  &]       &m &$\}$\cr  \noalign{\hrule}
      {} 1 1 1 0&SO &RS &.      &$>$&N  &\char94 &n &\char126\cr
                    \noalign{\hrule}
      {} 1 1 1 1&SI &US &/      &?  &O  &$\_$    &o &DEL\cr\noalign{\hrule}
   \noalign{\vskip2mm}
          \omit&\omit\namefont \hfil LEGEND:\hfil \span\omit
          &\multispan4\hskip\fatlinewidth
            \vtop{\vskip-10pt\hbox{\vrule
             \vbox to \rowheight{
                 \offinterlineskip
                 \hrule\vskip \topwhite
                 \hbox to \colwidth{\codefont\hskip\leftwhite
                                       dec\hfil}
                 \vfil
                 \hbox to \colwidth{\namefont\hfil CHAR\hfil}
                 \vfil
                 \hbox to \colwidth{\codefont\hskip\leftwhite
                                       hex\hfil oct
                                       \hskip\rightwhite}
                 \vskip\botwhite
                 \hrule                               }%
                                  \vrule}}
                      \hfil
          &\multispan2\bitfont \hskip\fatlinewidth
                      \vtop{\vskip-8pt\baselineskip=8.5pt
                      		% On instruction of Victor
                            \hbox{Victor Eijkhout}
                            \rlap{TACC}
                            %\rlap{University of Tennessee}
                            \rlap{Austin, Texas, USA}
                            }\hfil\cr
          }
}
%%%%%%%%%%%%%%%%%%%%%%% and that's it folks! %%%%%%%%%%%%%%%%%%%%%%%%%%

%\bye          %% PLAIN TeX
%\end{document} %% LaTeX
}

%\bigskip\bigskip
%{\centering\small Deze tabel kan gevonden worden op \url{http://www.ctan.org/tex-archive/info/ascii-chart}}
\end{table}

De naam ASCII betekent \textsl{American Standard Code for Information Interchange} en
dat geeft al goed aan waarvoor de code bedoeld is: op een gestandaardiseerde wijze
informatie uitwisselen. De code is in 1963 voor het eerst gepubliceerd~\cite{asa1963ascii}
en in die tijd
was er nog geen noodzaak om andere tekens te gebruiken dan de bekende westerse letters,
cijfers en leestekens. Vandaar dat het aantal tekens beperkt is.

De ASCII-code bestaat uit 128 7-bits tekens zoals te zien is in
tabel~\ref{tab:talascii-code}. De tekens zijn verdeeld in leesbare tekens, zoals letters,
cijfers en leestekens en zogenoemde \textsl{besturingstekens}. De besturingstekens zijn nodig
\index{besturingsteken}
om informatie-overdracht af te bakenen en om een bepaalde \textsl{handshake} 
(uitwisselingsprotocol) te regelen.
Zo zijn er codes voor de \textsl{carriage return} (CR, code~$13_{10}$) en de
\textsl{backspace} (BS, code $8_{10}$). De eerste 32 tekens zijn besturingstekens
waarvan de meeste tegenwoordig niet meer gebruikt worden~\cite{maini2007digital}.

De codes zijn niet willekeurig toegekend dat we goed kunnen zien bij de cijfers
en letters. De tabel is zo opgesteld dat de cijfers elkaar opvolgen. Dat is
handig bij het afdrukken van een (decimaal) getal.
Hetzelfde geldt voor de letters, ook die volgen elkaar op. De makers hebben ook nagedacht
over de positie van hoofd- en kleine letters. Deze verschillen in de tabel in slechts
\'{e}\'{e}n bit (bit b$_{6}$ in de tabel). Bij het gebruik van de Caps Lock-toets
of Shift-toets hoeft dus maar \'{e}\'{e}n bit (in combinatie met een letter) gewijzigd
te worden.

Een aantal besturingstekens is ook in C direct te gebruiken. Ze worden in C
\textsl{escape sequences} genoemd.\index{escape sequence}. Een escape sequence
begint altijd met een \textsl{backslash} (`\textbackslash')\index{backslash}
gevolgd door een letter, cijfer of teken. Zo is het teken voor een horizontale
tab `\texttt{\textbackslash t}' en voor de \textsl{line feed} is het teken
`\texttt{\textbackslash n}'. Met een line feed gaat de cursor naar het begin van
de volgende regel. De code \texttt{SP} staat voor één spatie.

