\addtocontents{toc}{\protect\newpage}
\chapter{Pointers}
\label{cha:pointers}
\thispagestyle{empty}

Een pointer-variabele, of kortweg \textsl{pointer}\index{pointer}, is een variabele waarvan de inhoud het adres is van een andere variabele. We zeggen dan ook wel dat de pointer \textsl{wijst} (Engels: ``points to'') naar de andere variabele. Als een pointer naar een variabele wijst, is het mogelijk om via de pointer bij de variabele te komen.

Pointers zijn een krachtig middel om efficiënt gegevens te beheren en parameters over te dragen aan functies. Soms zijn pointers zelfs de enige manier voor het bewerken van data. Het is dan ook niet verwonderlijk dat in veel C-programma's pointers gebruikt worden. 
%We zullen zien dat pointers en array's nauw met elkaar verbonden zijn.

Pointers worden samen met het \texttt{goto}-statement in verband gebracht met het schrijven van ondoorzichtige programma's. En zeker, onzorgvuldig gebruik van pointers komt de leesbaarheid en aantonen van correctheid van programma's niet ten goede. Maar met discipline kunnen programma's zeer efficiënt geschreven worden.

Een handige manier om pointers weer te geven, is door het geheugen van een computer voor te stellen als een rij vakjes. In figuur~\ref{fig:poiinmem} is dat te zien. Elk vakje stelt een geheugenplaats voor\footnote{We gaan hier gemakshalve vanuit dat elke variabele precies één geheugenplaats in beslag neemt. In de praktijk bestaan variabelen en pointers meestal uit meer dan één geheugenplaats.}. In de figuur zijn de variabelen \texttt{i} en \texttt{j} te zien. De twee pointers \texttt{p} en \texttt{q} wijzen respectievelijk naar variabele \texttt{i} en \texttt{j}. Dit is weergegeven met de twee pijlen. De inhoud van pointer \texttt{p} is dus het adres van variabele \texttt{i} en de inhoud van pointer \texttt{q} is het adres van variabele \texttt{j}.

%picture is a bit to high...
\vspace*{-0.5\baselineskip}
\begin{figure}[!ht]
\centering
\begin{tikzpicture}[pointerstyle]
\foreach \ii [count=\i from 0] in {1,...,15} {
	\node[memlocarray] (nod\i) at (\i*\unitsize,0) {};
}
\draw[center*-latex,shorten >=2pt] (nod2.center) to [bend left] (nod10.north);
\draw (nod2.center) node [yshift=-\unitsize cm] {p};
\draw[center*-latex,shorten >=2pt] (nod4.center) to [bend right] (nod13.south);
\draw (nod4.center) node [yshift=-\unitsize cm] {q};
\draw (nod10.center) node [yshift=-\unitsize cm] {i};
\draw (nod13.center) node [yshift=-\unitsize cm] {j};
\end{tikzpicture}
\caption{Uitbeelding van twee pointers naar variabelen in het geheugen wijzen.}
\label{fig:poiinmem}
\end{figure}

Een pointer kan wijzen naar een enkelvoudige variabele, een (element van een) array, een structure of (het begin van) een functie. Een pointer kan niet wijzen naar een constante, een uitdrukking en een \texttt{register} variabele.

Het is ook mogelijk om een pointer naar het datatype \texttt{void} te laten ``wijzen''.
Deze pointers worden generieke pointers genoemd. We zullen dit bespreken in paragraaf~\ref{sec:pointertovoid}.

\section{Pointers naar enkelvoudige datatypes}
\label{sec:pointersnaarenkelvoudigedatatypes}
In listing~\ref{cod:poidecla} is de declaratie van enkele pointers van enkelvoudige datatypes te zien. Bij de declaratie moet het type variabele waarnaar de pointer wijst worden opgegeven. De asterisk (\texttt{*}) geeft aan dat het de declaratie van een pointer betreft.

\begin{figure}[!ht]
\begin{lstlisting}[caption=Enkele declaraties van pointers.,label=cod:poidecla]
int *pint;          /* pint is a pointer to an int */
char *pchar;        /* pchar is a pointer to a character */
double *pdouble;    /* pdouble is a pointer to a double */
\end{lstlisting}
\end{figure}

We kunnen pointers uitbeelden door middel van vakjes, zoals te zien is in figuur~\ref{fig:poivoorstelling1}. Elk vakje stelt een pointer voor. In beginsel hebben de pointers geen correcte inhoud. Er wordt dan wel gesproken dat de pointer nergens naar toe wijst, maar dat is feitelijk onjuist. Een pointer heeft altijd een adres als inhoud, maar het kan zijn dat de pointer niet naar een bekende variabele wijst.

\begin{figure}[!ht]
\centering
\begin{tikzpicture}[pointerstyle]
\node[memloc] (A) at (0,0) {};
\draw (A) node [yshift=\unitsize cm] {pint};
\node[memloc] (B) at (4,0) {};
\draw (B) node [yshift=\unitsize cm] {pchar};
\node[memloc] (C) at (8,0) {};
\draw (C) node [yshift=\unitsize cm] {pdouble};
\end{tikzpicture}
\caption{Voorstelling van drie pointers in het geheugen.}
\label{fig:poivoorstelling1}
\end{figure}

Aan een pointer is het adres van een variabele van hetzelfde type toe te kennen. Hiervoor gebruiken we de \textsl{adres-operator} \texttt{\&} (ampersand)\indextwo{\&}{adres}. In listing~\ref{cod:poiassign} is een aantal toekenningen van adressen te zien.

\begin{figure}[H]
\begin{lstlisting}[caption=Enkele toekenningen van adressen aan pointers.,label=cod:poiassign]
int i;           /* the variables */
char c;
double d;

int *pint;       /* the pointers */
char *pchar;
double *pdouble;

pint = &i;       /* pint points to variable i */
pchar = &c;      /* pchar points to variable c */
pdouble = &d;    /* pdouble points to variable d */
\end{lstlisting}
\end{figure}

Nu de pointers geïnitialiseerd zijn, kunnen we een voorstelling maken van de relatie tussen de pointers en de variabelen. Dit is te zien in figuur~\ref{fig:poivoorstelling2}. Pointer \texttt{pint} wijst naar variabele \texttt{i}, pointer \texttt{pchar} wijst naar variabele \texttt{c} en pointer \texttt{pdouble} wijst naar variabele \texttt{d}.

\begin{figure}[!ht]
\centering
\begin{tikzpicture}[pointerstyle]
\node[memloc] (A) at (0,0) {};
\draw (A) node [yshift=\unitsize cm] {pint};
\node[memloc] (Aa) at (4,0) {};
\draw (Aa) node [yshift=\unitsize cm] {i};
\draw[center*-latex] (A.center) -- (Aa.west);

\node[memloc] (A) at (0,-2) {};
\draw (A) node [yshift=\unitsize cm] {pchar};
\node[memloc] (Aa) at (4,-2) {};
\draw (Aa) node [yshift=\unitsize cm] {c};
\draw[center*-latex] (A.center) -- (Aa.west);

\node[memloc] (A) at (0,-4) {};
\draw (A) node [yshift=\unitsize cm] {pdouble};
\node[memloc] (Aa) at (4,-4) {};
\draw (Aa) node [yshift=\unitsize cm] {d};
\draw[center*-latex] (A.center) -- (Aa.west);
\end{tikzpicture}
\caption{Voorstelling van drie pointers die naar variabelen wijzen.}
\label{fig:poivoorstelling2}
\end{figure}

Via de pointers kunnen we de variabelen gebruiken. Stel dat we variabele \texttt{i} met één willen verhogen. Dat kunnen we doen door gebruik te maken van de \textsl{indirectie} of \textsl{dereferentie} operator \texttt{*}\indextwo{*}{dereferentie}. Deze operator heeft voorrang op de rekenkundige operatoren.

\begin{figure}[!ht]
\begin{lstlisting}[caption=Gebruik van een pointer bij een toekenning.]
*pint = *pint + 1;            /* increment i by one */
*pchar = *pchar + 1;          /* next characterin  ASCII table */
*pdouble = *pdouble + 1.0;    /* add 1.0 to double */
\end{lstlisting}
\end{figure}

We mogen de dereferentie operator overal gebruiken waar een variabele gebruikt mag worden, bijvoorbeeld bij een optelling of in een test.

\begin{figure}[!ht]
\begin{lstlisting}[caption=Gebruik van een pointer bij het afdrukken van een variabele.,escapeinside={}{}]
int i = 2, *pint;
...
pint = &i;
...
i = *pint + 1;    /* add 1 to variable i */
...
if (*pint > 5) {
    printf("Variabele is %d\n", *pint);
}
\end{lstlisting}
\end{figure}

Overigens kan tijdens declaratie ook gelijk de initialisatie van een pointer plaatsvinden. Deze declaratie kan verwarrend zijn. In onderstaande listing wordt de pointer \texttt{pint} gedeclareerd en geïnitialiseerd met het adres van variabele \texttt{i}. Het betreft hier dus \textsl{geen} dereferentie.

\begin{figure}[!ht]
\begin{lstlisting}[caption=Declaratie en initialisatie van een pointer.]
int i = 2;
int *pint = &i;     /* declare and initialize pint */
\end{lstlisting}
\end{figure}

\section{De NULL-pointer}
\label{sec:nullpointer}
\index{NULL-pointer}\index{pointer!NULL}
In principe wijst een pointer naar een variabele (of beter: naar een geheugenadres). Om aan te geven dat een pointer niet naar een variabele wijst, kunnen we de \textsl{NULL-pointer} gebruiken. Let erop dat de NULL-pointer niet hetzelfde is als een niet-geïnitialiseerde pointer. Een NULL-pointer is een pointer waarin alle bits 0 zijn\footnote{In het algemeen is de inhoud van een NULL-pointer het getal 0, maar dat is niet in alle gevallen zo. De C-standaard definieert de NULL-pointer als een pointer die niet naar een bekende variabele wijst.}. Een niet-geïnitialiseerde pointer heeft een willekeurige waarde\footnote{In geval een pointer als globale variabele wordt gedeclareerd, zorgt de compiler ervoor dat de inhoud op 0 gezet wordt.}. In C is een preprocessor-macro genaamd \texttt{NULL} te gebruiken om een pointer als NULL-pointer te initialiseren. De C-standaard schrijft voor voor dat \texttt{NULL} wordt gedefinieerd in \texttt{locale.h}, \texttt{stddef.h}, \texttt{stdio.h}, \texttt{stdlib.h}, \texttt{string.h}, \texttt{time.h}, en \texttt{wchar.h}. Slechts een van deze header-bestanden is noodzakelijk om \texttt{NULL} te definiëren.

\begin{figure}[!ht]
\begin{lstlisting}[caption=Declaratie en initialisatie van een NULL-pointer.]
#include <stdio.h>

int *p = NULL;   /* p is initialized as NULL-pointer */
\end{lstlisting}
\end{figure}

NULL-pointers kunnen \textsl{niet} gebruikt worden bij dereferentie. Dat veroorzaakt over het algemeen dat de executie van een programma wordt afgebroken.

\begin{figure}[!ht]
\begin{lstlisting}[caption=Dereferentie van een NULL-pointer.]
#include <stdio.h>

int *p = NULL;   /* p is initialized as NULL-pointer */

*p = *p + 1;     /* Oops, dereference of NULL-pointer! */
\end{lstlisting}
\end{figure}

In een vergelijking zullen twee NULL-pointers altijd \texttt{true} opleveren.

\begin{figure}[H]
\begin{lstlisting}[caption=Vergelijken van twee NULL-pointers.]
int *p = NULL, *q = NULL;
...
if (p == q) { /* true */ }
\end{lstlisting}
\end{figure}

De NULL-pointer wordt door diverse standaard functies gebruikt om aan te geven dat er een fout is geconstateerd. Zo geeft de functie \texttt{fopen} de waarde \texttt{NULL} terug als het niet gelukt is om een bestand te openen. De functie \texttt{malloc} geeft de waarde \texttt{NULL} terug als het niet gelukt is om een stuk geheugen te alloceren.

Bij het voorstellen van NULL-pointers zijn diverse mogelijkheden die gebruikt worden. Bij de linker voorstelling wordt het woord \texttt{NULL} in een vakje gezet, bij de middelste voorstelling wordt een kruis in het vakje gezet en bij de rechter voorstelling wordt een pijl getrokken naar een vakje met een kruis erin.

\begin{figure}[!ht]
\centering
\begin{tikzpicture}[pointerstyle]
\node (null1) at (0,0) [memlocnull] {};
\draw (null1) node [yshift=\unitsize cm] {p};
\node (null2) at (3,0) [memloc] {};
\draw (null2) node [yshift=\unitsize cm] {p};
\node (null20) at (5,0) [memlocnull] {};
\draw[center*-latex] (null2.center) -- (null20.west);
\node (null3) at (-3,0) [memloc] {NULL};
\draw (null3) node [yshift=\unitsize cm] {p};
\end{tikzpicture}
\caption{Drie voorstellingen van NULL-pointers.}
\label{fig:poinullpointers}
\end{figure}


\section{Pointer naar \texttt{void}}
\index{void-pointer}\index{pointer!naar void}
\label{sec:pointertovoid}
%The void pointer, also known as the generic pointer, is a special type of pointer that can be pointed at objects of any data type! A void pointer is declared like a normal pointer, using the void keyword as the pointer’s type:

De void-pointer, ook wel \textsl{generieke pointer}\index{generieke pointer} genoemd, is een speciaal type pointer die naar elk type variabele kan wijzen. Een void-pointer wordt net als een gewone pointer gedeclareerd middels het keyword \texttt{void}. Toekenning aan een void-pointer gebeurt met de adres-operator~\texttt{\&}.

\begin{figure}[!ht]
\begin{lstlisting}[caption=Declaratie en initialisatie van een void-pointer.]
int i;
char c;
double d;

void *p;  /* void-pointer */

p = &i;   /* valid */
p = &c;   /* valid */
p = &d;   /* valid */
\end{lstlisting}
\end{figure}

Omdat het type van een void-pointer niet bekend is, kan een void-pointer niet zonder meer in een dereferentie gebruikt worden. De void-pointer moet expliciet gecast worden naar het correcte type. In de onderstaande listing gebruiken we pointer \texttt{p} om naar een \texttt{int} te wijzen. De type cast \texttt{(int *)} zorgt ervoor dat pointer \texttt{p} naar een \texttt{int} wijst. Door gebruik te maken van de dereferentie operator \texttt{*} kunnen we bij de inhoud van variabele \texttt{i}. De constructie \texttt{*(int *)} is dus een expliciete dereferentie naar een integer.

\begin{figure}[!ht]
\begin{lstlisting}[caption={Declaratie, initialisatie en deference van een void-pointer.}]
int i = 2;
void *p = &i;                    /* void-pointer */

...
*(int *) p = *(int *) p + 1;     /* explicit type cast */

...
printf("De waarde is %d\n", *(int *) p);
\end{lstlisting}
\end{figure}


\section{Afdrukken van pointers}
\label{sec:afdrukkenvanpointers}
\index{pointer!afdrukken van}
Het afdrukken van de waarde van een pointer kan met de \texttt{printf}-functie en de format specifier \texttt{\%p}. Merk op dat de pointer van het type \texttt{void} moet zijn, maar veel compilers accepteren pointers naar een datatype.

\begin{figure}[!ht]
\begin{lstlisting}[caption=Afdrukken van een pointer.]
#include <stdio.h>

int main() {

    int i = 2, *p = &i;

    printf("Pointer: %p\n", (void *) p);

    return 0;
}
\end{lstlisting}
\end{figure}

Noot: Bij 32-bits compilers is de grootte van een pointer 32 bits (4 bytes). Zo'n pointer kan maximaal 4 GB adresseren. Bij 64-bits compilers is de grootte van een pointer 64 bits (8 bytes). Zo'n pointer kan maximaal 16 EB (exa-bytes) adresseren.


\section{Pointers naar array's}
\label{sec:pointersnaararrays}
\index{pointer!naar array}
We kunnen een pointer ook laten wijzen naar het eerste element van een array. Dit is te zien in listing~\ref{cod:poifirstarray}. De array bestaat uit negen elementen van het type \texttt{int}. De pointer \texttt{p} laten we wijzen naar het eerste element van de array. We gebruiken hiervoor de adres-operator \texttt{\&} en elementnummer 0.

\begin{figure}[!ht]
\begin{lstlisting}[caption=Een pointer naar het eerste element van een array.,label=cod:poifirstarray]
int ary[] = {3,6,1,0,9,7,6,2,7};

int *p = &ary[0]; /* p points to first element for array */
\end{lstlisting}
\end{figure}

De uitbeelding hiervan is te zien in figuur~\ref{fig:poifirstarray}.
%
\begin{figure}[!ht]
\centering
\begin{tikzpicture}[pointerstyle]
\foreach \ii [count=\i from 0] in {3,6,1,0,9,7,6,2,7} {
	\node[memlocarray] (nod\i) at (\i*\unitsize,0) {\ii};
	\draw (nod\i) node[yshift=\unitsize cm] {\footnotesize\i}; 
}
\node at (-0.5,0) [left] {ary};
\node (A) at (-2,-2) [memloc] {};
\draw (A) node [yshift=\unitsize cm] {p};
\draw[center*-latex] (A.center) to [bend right] (nod0.south);
\end{tikzpicture}
\caption{Uitbeelding van een pointer naar het eerste element van een array.}
\label{fig:poifirstarray}
\end{figure}
%
Omdat deze toekenning zeer vaak in een C-programma voorkomt, is er een verkorte notatie mogelijk. We kunnen in plaats van \texttt{\&ary[0]} ook de naam van de array gebruiken: \textsl{de naam van een array is een synoniem voor een adres van het eerste element van de array}. Zie listing~\ref{cod:poifirstarray2}.

\begin{figure}[!ht]
\begin{lstlisting}[caption=Een pointer naar het eerste element van een array.,label=cod:poifirstarray2]
int ary[] = {3,6,1,0,9,7,6,2,7};

int *p = ary; /* p points to first element of array */
\end{lstlisting}
\end{figure}

Omdat de naam van een array een synoniem is, mag het dus niet gebruikt worden aan de linkerkant van een toekenning.

\begin{figure}[!ht]
\begin{lstlisting}[caption=Een pointer naar het eerste element van een array.,label=cod:poifirstarray3]
int *p, ary[] = {3,6,1,0,9,7,6,2,7};

p = ary;     /* correct use of pointer and array name */
ary[2] = *p; /* correct use of pointer and array element */

ary = p;     /* ERROR: array name cannot be used in this context */
\end{lstlisting}
\end{figure}

Het is ook mogelijk om een pointer naar een ander element van een array te laten wijzen. De compiler test niet of de toekenning binnen de array-grenzen ligt.

\begin{figure}[!ht]
\begin{lstlisting}[caption=Een pointer naar het derde element van een array.,label=cod:poithirdarray]
int ary[] = {3,6,1,0,9,7,6,2,7};

int *p = &ary[2]; /* p points to third element of array */
\end{lstlisting}
\end{figure}

Een uitbeelding is te zien in figuur~\ref{fig:poithirdarray}. In de figuur wijst \texttt{p} naar het derde element van \texttt{ary}. We kunnen nu de inhoud van dit element opvragen door \texttt{ary[2]} en door \texttt{*p}. Het is zelfs mogelijk om \texttt{p} als de naam van een array te beschouwen. Zie paragraaf~\ref{sec:relatietussenpointersenarrays}.

\begin{figure}[!ht]
\centering
\begin{tikzpicture}[pointerstyle]
\foreach \ii [count=\i from 0] in {3,6,1,0,9,7,6,2,7} {
	\node[memlocarray] (nod\i) at (\i*\unitsize,0) {\ii};
	\draw (nod\i) node[yshift=\unitsize cm] {\footnotesize\i}; 
}
\node at (-0.5,0) [left] {ary};

\node (A) at (-1,-2) [memloc] {};
\draw (A) node [yshift=\unitsize cm] {p};
\draw[center*-latex] (A.center) to [out=0, in=-90] (nod2.south);
\end{tikzpicture}
\caption{Uitbeelding van een pointer naar het derde element van een array.}
\label{fig:poithirdarray}
\end{figure}

We kunnen de lengte van \texttt{ary} bereken met de \texttt{sizeof}-operator. Dat kan alleen via \texttt{ary} omdat de compiler de lengte kan uitrekenen. Het kan \textsl{niet} via pointer \texttt{p} want dat is een pointer naar een \texttt{int}; pointer \texttt{p} ``weet'' niet dat er naar een array gewezen wordt.


\section{Strings}
\label{sec:strings}
\index{string}
Een string in C is niets anders dan een array van karakters, afgesloten met een nul-karakter\index{nul-karakter}\indextwo{\textbackslash 0}{nul-karakter} (\lstinline|'\0'|). Er is dus altijd één geheugenplaats meer nodig dan het aantal karakters in de string. Een nul-karakter is niet hetzelfde als een NULL-pointer. Een nul-karakter is een byte met de inhoud 0 (alle bits zijn 0), een NULL-pointer is een pointer met de inhoud 0. in listing~\ref{cod:poistrings} zijn twee declaraties met strings te zien, een echte array met en string als inhoud en een pointer naar een string in het geheugen.

\begin{figure}[!ht]
\begin{lstlisting}[caption=Declaratie en initialisatie van twee C-strings.,label=cod:poistrings]
char str[] = "Hello ";
char *pstr = "world!";
\end{lstlisting}
\end{figure}

Merk op dat \texttt{str} niet aangepast mag worden, want dit is de naam van een array. Pointer \texttt{pstr} mag wel aangepast worden want \texttt{pstr} is een pointer naar het eerste element van de array. Een voorstelling van beide strings is te zien in figuur~\ref{fig:poistrings}.

\begin{figure}[!ht]
\centering
\begin{tikzpicture}[pointerstyle]
\foreach \ii [count=\i from 0] in {h, e, l, l, o, \ , \textbackslash 0} {
	\node[memlocarray] (nod\i) at (\i*\unitsize,0) {\ii};
	\draw (nod\i) node[yshift=\unitsize cm] {\footnotesize\i}; 
}
\node at (-0.5,0) [left] {str};

\foreach \ii [count=\i from 0] in {w, o, r, l, d, ! , \textbackslash 0} {
	\node[memlocarray] (nod\i) at (\i*\unitsize,-2) {\ii};
	\draw (nod\i) node[yshift=\unitsize cm] {\footnotesize\i}; 
}

\node (A) at (-3,-2) [memloc] {};
\draw (A) node [yshift=\unitsize cm] {pstr};
\draw[center*-latex] (A.center) -- (nod0.west);
\end{tikzpicture}
\caption{Uitbeelding van twee C-strings.}
\label{fig:poistrings}
\end{figure}

Ook hier merken we op dat we de lengte van \texttt{str} kunnen berekenen met de \texttt{sizeof}-operator. De compiler heeft genoeg informatie beschikbaar. We kunnen de lengte van de tweede string \textsl{niet} door de compiler laten uitrekenen, want \texttt{pstr} in een pointer naar een \texttt{char}. Pointer \texttt{pstr} ``weet'' dus niet dat er naar een string gewezen wordt.

Toch is het mogelijk om tijdens het draaien van een programma de lengte van de string te vinden. We kunnen namelijk uitgaan van het feit dat een string in `C` wordt afgesloten met een nul-karakter. Dit wordt uitgelegd in de volgende paragraaf.


\section{Rekenen met pointers}
\label{sec:rekenenmetpointers}
\index{pointer!rekenen met}
Pointers kunnen rekenkundig worden aangepast dat vooral nuttig is bij het gebruik van array's.
In het onderstaande programma wijst pointer \texttt{p} in eerste instantie naar het begin van de array \texttt{ary} (dus \texttt{ary[0]}). Daarna wordt \texttt{p} twee maal met 1 verhoogd en daarna met 3 verhoogd. Bij rekenkundige operaties op pointers wordt rekening gehouden met de grootte van de datatypes. De grootte van een \texttt{int} is in de regel vier bytes. Door de pointer met 1 te verhogen wordt dus naar de volgende \texttt{int} gewezen.

\begin{figure}[!ht]
\begin{lstlisting}[caption=Rekenen met pointers.]
int ary[] = {3,6,1,0,9,7,6,2,7};
int *p = ary;  /* p pointe to ary[0] */

p = p + 1;     /* p points to ary[1] */
...
p = p + 1;     /* p points to ary[2] */
...
p = p + 3;     /* p points to ary[5] */

\end{lstlisting}
\end{figure}

Een mooi voorbeeld van het rekenen met pointers is het bepalen van de lengte van een C-string. In listing~\ref{cod:poistrlen1} wordt pointer \texttt{str} gedeclareerd en wijst naar het begin van de string. Pointer \texttt{begin} wijst ook naar het begin van de string. Daarna verhogen we pointer \texttt{str} totdat het einde van de string is bereikt. Daarna drukken we het verschil van de twee pointers af.

\begin{figure}[!ht]
\begin{lstlisting}[caption=Berekenen van de lengte van een string met behulp van pointers.,label=cod:poistrlen1,escapeinside={}{}]
#include <stdio.h>

int main() {

    char *str = "Hallo wereld!";
    char *begin = str;

    while (*str != '\0') { /* while not end of string ... */
        str = str + 1;     /* point to the next character */
    }

    printf("Lengte is %d\n", str-begin);
}
\end{lstlisting}
\end{figure}

Let erop dat de twee pointers naar elementen in dezelfde array moeten wijzen (of één na het laatste element). Alleen dan levert de aftrekking \texttt{str-begin} een gedefinieerd resultaat. De aftrekking is van het type \texttt{ptrdiff\_t} (dat meestal gelijk is aan een \texttt{int}) en levert het verschil in elementen. Het onderstaande programmafragment geeft als uitvoer de waarde~3.

\begin{figure}[!ht]
\begin{lstlisting}[caption=Het berekenen van het verschil van twee pointers.]
int ary[] = {3,6,1,0,9,7,6,2,7};
int *p = &ary[2];
int *q = &ary[5];

printf("Verschil is %d\n", q-p);
\end{lstlisting}
\end{figure}

Vergelijken van twee pointers kan ook. Zo kunnen pointers op gelijkheid worden vergeleken, maar ongelijkheid kan ook. We zouden de \texttt{printf}-regel van listing~\ref{cod:poistrlen1} kunnen vervangen door de onderstaande programmafragment.
Uiteraard moeten de twee pointers naar hetzelfde datatype wijzen en heeft de vergelijking alleen zin als de pointers naar elementen binnen dezelfde array wijzen.

\begin{figure}[!ht]
\begin{lstlisting}[caption=Vergelijken van twee pointers.]
    if (str>begin) {
        printf("Lengte is %d\n", str-begin);
    } else {
        printf("De string is leeg\n");
    }
\end{lstlisting}
\end{figure}


\section{Relatie tussen pointers en array's}
\label{sec:relatietussenpointersenarrays}
De relatie tussen pointers en array's zijn zo sterk in `C' verankerd, dat we er een aparte paragraaf aan wijden. In listing~\ref{cod:poiarrayandpointer} zijn de declaratie en initialisatie van een array en een pointer te zien. De pointer \texttt{p} wijst na initialisatie naar het eerste element van de array.

\begin{figure}[!ht]
\begin{lstlisting}[caption=Declaratie en initialisatie van een array en een pointer.,label=cod:poiarrayandpointer]
int ary[] = {3,6,1,0,9,7,6,2,7};
int *p = ary;
\end{lstlisting}
\end{figure}

Om toegang te krijgen tot eerste element uit de array kunnen natuurlijk \texttt{ary[0]} gebruiken. Maar we kunnen via \texttt{p} ook bij het eerste element komen. Hiervoor gebruiken we \texttt{*p}. We mogen echten \texttt{p} ook lezen als de naam van een array. Om het het eerste element te komen, mogen we dus ook \texttt{p[0]} gebruiken. Om alle elementen van de array bij elkaar op te tellen, kunnen we dus schrijven:

\begin{figure}[!ht]
\begin{lstlisting}[caption=Bepalen sum van elementen in een array.,label=cod:poiarrayandpointersym]
int ary[] = {3,6,1,0,9,7,6,2,7};
int *p = ary;

int sum = p[0]+p[1]+p[2]+p[3]+p[4]+p[5]+p[6]+p[7]+p[8];
\end{lstlisting}
\end{figure}

Aan de andere kant mogen we de naam van de array ook lezen als een pointer naar het eerste element. We kunnen de naam gebruiken in een dereference. Dat betekent dat \texttt{*ary} identiek is aan \texttt{ary[0]} en dat \texttt{*(ary+2)} identiek is aan \texttt{ary[2]}. We hebben echter wel haken nodig bij \texttt{*(ary+2)} omdat de dereferentie-operator voorgaat op de optelling. Om de som van de array te bepalen mogen we dus schrijven:

\begin{figure}[!ht]
\begin{lstlisting}[caption=Bepalen sum van elementen in een array.,label=cod:poiarrayandpointersym2]
int ary[] = {3,6,1,0,9,7,6,2,7};
int *p = ary;

int sum = *ary + *(ary+1) + *(ary+2) + *(ary+3) + ... ;
\end{lstlisting}
\end{figure}

Het is mogelijk om een pointer naar een willekeurig element van de array te laten wijzen door de naam van de array als pointer te beschouwen. Als we \texttt{p} willen laten wijzen naar het derde element gebruiken we gewoon de toekenning \texttt{p = ary+2}. Na deze toekenning kunnen het derde element afdrukken door \texttt{p} als pointer of als array te beschouwen. Zie listing~\ref{cod:poipointstoelement}.

\begin{figure}[!ht]
\begin{lstlisting}[caption=Pointer die naar een element in een array wijst.,label=cod:poipointstoelement]
int ary[] = {3,6,1,0,9,7,6,2,7};

int *p = ary+2;                    /* p points to third element */

printf("Contents is %d\n", *p);    /* prints third element */
printf("Contents is %d\n", p[0]);  /* prints third element */
\end{lstlisting}
\end{figure}

Let erop dat we in het bovenstaande programmafragment \texttt{p[0]} hebben gebruikt. De pointer wijst naar het derde element dus \texttt{p[0]} betekent dat de inhoud van het derde element wordt afgedrukt.

Als we zeker weten dat we een correct element gebruiken, mogen we ook negatieve waarden voor het elementnummer gebruiken. In listing~\ref{cod:poipointstonegeindex} wordt het adres van het derde element uit de array toegekend aan pointer \texttt{p}. We mogen dan \texttt{p[-1]} gebruiken omdat dit het tweede element uit de array betreft. We kunnen echter niet \texttt{ary[-1]} gebruiken want dit leidt tot het gebruik van een element buiten de array. Let erop dat C-compilers over het algemeen niet testen of een adressering binnen de array-grenzen ligt.

\begin{figure}[!ht]
\begin{lstlisting}[caption=Pointer die naar een element in een array wijst.,label=cod:poipointstonegeindex]
int ary[] = {3,6,1,0,9,7,6,2,7};

int *p = ary+2;    /* p points to third element */

int a = p[-1];     /* legal: points to second element */
int b = ary[-1];   /* ILLEGAL: points outside array */
\end{lstlisting}
\end{figure}


\section{Pointers als functie-argumenten}
\label{sec:pointersalsfunctieargumenten}
\index{pointer!als functie-argument}
Net als ``gewone'' variabelen, kunnen ook pointers als argumenten bij het aanroepen van een functie gebruikt worden. Let erop dat een kopie van de pointers worden meegegeven. Via die kopie kunnen we bij de variabelen komen waar de pointers naartoe wijzen. We kunnen dus niet de pointers zelf aanpassen.

In listing~\ref{cod:poiswaptwo} is te zien hoe we een functie \texttt{swap} definiëren die de inhouden van twee variabelen verwisseld. Bij het aanroepen van de functie geven we de addressen mee van de te verwisselen variabelen. In de functie gebruiken we pointers om de inhouden te verwisselen.

\begin{figure}[!ht]
\begin{lstlisting}[caption=Het verwisselen van twee variabelen met behulp van pointers.,label=cod:poiswaptwo]
void swap(int *pa, int *pb) {
	int temp;

    temp = *pa;
    *pa = *pb;
    *pb = temp;
}
\end{lstlisting}
\end{figure}

Bij het aanroepen van de functie geven we de adressen van de variabelen mee. Dit is te zien in listing~\ref{cod:poiswaptwocalll}. Deze manier van argumentenoverdracht wordt \textsl{Call by Reference} genoemd.

\begin{figure}[!ht]
\begin{lstlisting}[caption=Aanroep van de functie.,label=cod:poiswaptwocalll]
void swap(int *pa, int *pb);   /* prototype of function swap */

int main(void) {

    int a=2, b=3;              /* declare variables */

    swap(&a, &b);              /* swap variables */

    return 0;
}
\end{lstlisting}
\end{figure}

Een typisch voorbeeld is een functie die een string kopieert naar een andere string. De functie krijgt twee pointers naar strings als argumenten mee. Bij het aanroepen van de functie worden de namen van de twee strings als argumenten meegegeven. De naam van een string is immers een pointer naar het eerste karakter van de string. Het programma is te zien in listing~\ref{cod:poicopystrings1}.

\begin{figure}[!ht]
\begin{lstlisting}[caption=Functie voor het kopieren van een string.,label=cod:poicopystrings1,escapeinside={}{}]
void string_copy(char *to, char *from) {

   if (from == NULL) {        /* sanity check */
       return;
   }

    while (*from != '\0') {   /* while not end of string ... */
        *to = *from;          /* copy character */
        to = to + 1;          /* point to the next character */
        from = from + 1;
    }
    *to = '\0';               /* terminate string */
}

int main() {

    char stra[] = "Hello world!";
    char strb[100];

    string_copy(strb, stra);  /* copy stra to strb */

	return 0;

}
\end{lstlisting}
\end{figure}

Merk op dat de geheugenruimte voor de kopie groot genoeg moet zijn om de kopie op te slaan en dat de twee string elkaar in het geheugen niet mogen overlappen. De standaard C-bibliotheek heeft een functie \texttt{strcpy} die een efficiënte implementatie is van het kopiëren van strings.

Noot: een array kan alleen maar via een pointer als argument aan een functie worden doorgegeven. Dit is veel efficiënter dan de hele array mee te geven. In listing~\ref{cod:poicopystrings1} wordt dus \textsl{niet} de hele array meegegeven, maar alleen de pointers naar de eerste elementen. We mogen daarom de pointers \texttt{to} en \texttt{from} ook als namen van array's beschouwen.  Zie listing~\ref{cod:poicopystrings2}. Merk op dat we dus de lengte van de meegegeven array \textsl{niet} kunnen uitrekenen met de \texttt{sizeof}-operator. Er wordt immers een pointer meegegeven. Dat we toch het einde van een string kunnen bepalen, komt doordat een string wordt afgesloten met een nul-byte.

\begin{figure}[!ht]
\begin{lstlisting}[caption=Functie voor het kopieren van een string.,label=cod:poicopystrings2]
void string_copy(char to[], char from[]) {

    int i=0;

    while (from[i] != '\0') {  /* while not end of string */
        to[i] = from[i];       /* ... copy character */
        i = i + 1;             /* point to next character */
    }
    to[i] = '\0';              /* .. and terminate string */
}
\end{lstlisting}
\end{figure}

Bij het overdragen van een array aan een functie moet expliciet de grootte worden opgegeven. Het is niet mogelijk om \textsl{in de functie} de grootte van de array uit te rekenen, er wordt immers een pointer meegegeven. Er is dus een extra parameter nodig waarmee we de grootte opgeven. Bij het aanroepen van de functie rekenen we de grootte uit en geven dit mee. Dit is te zien in listing~\ref{cod:poilengthpass}. Let erop dat in \texttt{main} w\'el de grootte van de array kan worden uitgerekend, want daar wordt de array gedeclareerd. We gebruiken twee keer de \texttt{sizeof}-operator, want \texttt{sizeof} geeft de grootte van een object in bytes. We moeten de grootte van de array in bytes delen door de grootte van \'e\'en element in bytes.5

\begin{figure}[!ht]
\begin{lstlisting}[caption=Meegeven van de grootte van een array.,label=cod:poilengthpass]
#include <stdio.h>
#include <stdlib.h>

void printarray(int ary[], int len) {

    int i;

    for (i=0; i<len; i++) {
        printf("%d ", ary[i]);
    }
}

int main(void) {

    int list[] = {1,2,3,4,5,6,7,8,9};
    int length = sizeof list / sizeof list[0];

    printarray(list, length);

    return 0;
}
\end{lstlisting}
\end{figure}

\begin{infobox}[Call by reference...]
\index{Call by Reference}
In de programmeerwereld is het gebruikelijk om onderscheid te maken tussen twee manieren van argumentenoverdracht: \textsl{Call by Value} en \textsl{Call by Reference}. Bij Call by Value wordt een kopie van de waarde van een variabele aan de functie meegegeven. Alleen de kopie kan veranderd worden door de functie. De variabele waarvan de kopie is gemaakt kan dus niet op deze manier veranderd worden. Bij Call by Reference wordt het adres van de variabele aan de functie meegegeven. Via dit adres is het dus wel mogelijk om de originele variabele te veranderen.

Sommige programmeurs beweren dat Call by Reference eigenlijk niet bestaat. En daar is wat voor te zeggen. Er wordt immers een waarde aan de functie meegegeven en dat is het adres van een variabele. Dit adres kan in de functie veranderd worden maar het originele adres waar de variabele in het geheugen staat wordt niet aangepast. Toch maken programmeurs onderscheid tussen deze twee manieren van argumentenoverdracht.
\end{infobox}


\section{Pointer als return-waarde}
\label{sec:pointersalsreturnwaarde}
\index{pointer!als return-waarde}
Een pointer kan ook als return-waarde dienen. In het onderstaande voorbeeld wordt in een string gezocht naar een bepaalde karakter in een string. Als het karakter gevonden is, wordt een pointer naar het karakter teruggegeven. Als het karakter niet wordt gevonden wordt \texttt{NULL} teruggegeven. Na het uitvoeren van de functie moet hier op getest worden. Tevens wordt in het begin getest of de pointer naar de string wel een geldige waarde heeft. Geldig wil zeggen dat de pointer niet \texttt{NULL} is. Het is \textsl{good practice} om altijd te testen of een pointer \texttt{NULL} is.

\begin{figure}[!ht]
\begin{lstlisting}[caption=Pointer als return-waarde.,escapeinside={}{}]
char *find_token(char *str, char ch) {

    if (str == NULL) {       /* sanity check */
        return NULL;
    }

    while (*str != '\0') {   /* while not end of string */
        if (*str == ch) {    /* if character found ... */
            return str;      /* return pointer */
        }
        str++;
    }

    return NULL;             /* character not found */
}
\end{lstlisting}
\end{figure}

Let goed op de definitie van de functie \texttt{find\_token}. Voor de functienaam is de dereferentie-operator geplaatst. Dit betekent dat de functie een pointer naar een karakter teruggeeft. Deze vorm wordt vaak verward met pointers naar functies. Zie paragraaf~\ref{sec:pointersnaarfunctie}.


\section{Pointers naar pointers}
\index{pointer!naar pointer}
\label{sec:pointersnaarpointers}
Een pointer kan ook gebruikt worden om naar een andere pointer te wijzen. In figuur~\ref{fig:poipointertopointer} is te zien dat pointer \texttt{p} wijst naar variabele \texttt{i}. Met behulp van de dereferentie \texttt{*p} kunnen we bij de inhoud van variabele \texttt{i} komen. De pointer \texttt{pp} wijst naar \texttt{p}. We hebben nu een \textsl{dubbele dereferentie}\index{dubbele dereferentie} de nodig om via pointer \texttt{pp} bij de inhoud van variabele \texttt{i} te komen. 

\begin{figure}[!ht]
\centering
\begin{tikzpicture}[pointerstyle]
\node (A) at (0,0) [memloc] {};
\node (B) at (3,0) [memloc] {};
\node (C) at (6,0) [memloc] {2};
\draw[center*-latex] (A.center) -- (B.west);
\draw[center*-latex] (B.center) -- (C.west);
\draw (A) node [yshift=\unitsize cm] {pp};
\draw (B) node [yshift=\unitsize cm] {p};
\draw (C) node [yshift=\unitsize cm] {i};
\end{tikzpicture}
\caption{Uitbeelding van een pointer naar een pointer naar een \texttt{int}.}
\label{fig:poipointertopointer}
\end{figure}

Voor de dubbele dereferentie gebruiken we twee keer de dereferentie-operator \texttt{*}. Om toegang te krijgen tot de inhoud van variabele \texttt{i} via pointer \texttt{pp} gebruiken we dus \texttt{**pp}. Dit is te zien in de onderstaande listing.

\begin{figure}[!ht]
\begin{lstlisting}[caption=Voorbeeld van een pointer naar een pointer.]
int i = 2;        /* the integer */
int *p = &i;      /* p points to i */
int **pp = &p;    /* pp points to p */

printf("De waarde is %d\n", **pp);
\end{lstlisting}
\end{figure}

Met behulp van pointers naar pointers kunnen de inhouden van twee pointers verwisselen. In figuur~\ref{fig:poipointertostring} is te zien dat de pointers \texttt{pa} en \texttt{pb} wijzen naar twee strings in het geheugen. Als we nu de strings willen ``verwisselen'', hoeven we alleen maar de pointers er naartoe te verwisselen.

\begin{figure}[!ht]
\centering
\begin{tikzpicture}[pointerstyle]
\foreach \ii [count=\i from 0] in {h, e, l, l, o, \ , \textbackslash 0} {
	\node[memlocarray] (nod\i) at (\i*\unitsize,0) {\ii};
	\draw (nod\i) node[yshift=\unitsize cm] {\footnotesize\i}; 
}
\node (A) at (-3,0) [memloc] {};
\draw (A) node [yshift=\unitsize cm] {pa};
\draw[center*-latex] (A.center) -- (nod0.west);

\foreach \ii [count=\i from 0] in {w, o, r, l, d, ! , \textbackslash 0} {
	\node[memlocarray] (nod\i) at (\i*\unitsize,-2) {\ii};
	\draw (nod\i) node[yshift=\unitsize cm] {\footnotesize\i}; 
}
\node (B) at (-3,-2) [memloc] {};
\draw (B) node [yshift=\unitsize cm] {pb};
\draw[center*-latex] (B.center) -- (nod0.west);
\end{tikzpicture}
\caption{Uitbeelding van pointers naar strings.}
\label{fig:poipointertostring}
\end{figure}

We kunnen dat doen met het onderstaande programmafragment. We declareren drie pointers en laten \texttt{pa} en \texttt{pb} wijzen naar strings. We gebruiken de pointer \texttt{temp} om de verwisseling tot stand te brengen.

\begin{figure}[H]
\begin{lstlisting}[caption=Verwisselen van twee pointers.]
char *pa = "hello ";
char *pb = "world!";
char *temp;

temp = pa;           /* swap pa and pb */
pa = pb;
pb = temp;
\end{lstlisting}
\end{figure}

Maar stel dat we zulke verwisselingen vaker in een programma moeten uitvoeren. Dan is het handig om een functie te gebruiken die dat voor ons doet. We geven aan de functie de adressen van de pointers mee zodat de functie ze kan verwisselen. Hoe dit eruit ziet, is te zien in figuur~\ref{fig:poipointertopointerstring}. De pointers \texttt{pa} en \texttt{pb} wijzen naar de strings. In de functie zijn twee pointers gedefinieerd die wijzen naar pointers naar strings. Dus \texttt{ppa} wijst naar pointer \texttt{pa} en \texttt{ppb} wijst naar pointer \texttt{pb}. We kunnen nu in de functie de inhouden van \texttt{pa} en \texttt{pb} verwisselen.

\begin{figure}[!ht]
\centering
\begin{tikzpicture}[pointerstyle]
\foreach \ii [count=\i from 0] in {h, e, l, l, o, \ , \textbackslash 0} {
	\node[memlocarray] (nod\i) at (\i*\unitsize,0) {\ii};
	\draw (nod\i) node[yshift=\unitsize cm] {\footnotesize\i}; 
}
\node (A) at (-3,0) [memloc] {};
\draw (A) node [yshift=\unitsize cm] {pa};
\draw[center*-latex] (A.center) -- (nod0.west);

\foreach \ii [count=\i from 0] in {w, o, r, l, d, ! , \textbackslash 0} {
	\node[memlocarray] (nod\i) at (\i*\unitsize,-2) {\ii};
	\draw (nod\i) node[yshift=\unitsize cm] {\footnotesize\i}; 
}
\node (B) at (-3,-2) [memloc] {};
\draw (B) node [yshift=\unitsize cm] {pb};
\draw[center*-latex] (B.center) -- (nod0.west);

\node (AA) at (-6,0) [memloc] {};
\draw (AA) node [yshift=\unitsize cm] {ppa};
\draw[center*-latex,dashed] (AA.center) -- (A.west);
\node (BB) at (-6,-2) [memloc] {};
\draw (BB) node [yshift=\unitsize cm] {ppb};
\draw[center*-latex,dashed] (BB.center) -- (B.west);
\end{tikzpicture}
\caption{Uitbeelding van pointers naar pointers naar strings.}
\label{fig:poipointertopointerstring}
\end{figure}

In listing~\ref{cod:poiswaptopointers} is de functie te zien voor het verwisselen van twee pointers. De functie heeft twee parameters \texttt{ppa} en \texttt{ppb} die een pointer zijn naar een pointer naar een string. In de functie declareren we een pointer \texttt{temp} die een pointer is naar string (of eigenlijk: een karakter). Met behulp van de dereferentie \texttt{*ppa} kopiëren we de inhoud van pointer \texttt{pa} naar \texttt{temp}. Daarna kopiëren we \texttt{pb} naar \texttt{pa} en als laatste kopiëren we \texttt{temp} naar \texttt{pb}.

\begin{figure}[!ht]
\begin{lstlisting}[caption=Functie voor het verwisselen van twee pointers.,label=cod:poiswaptopointers]
#include <stdio.h>
#include <stdlib.h>

void swapstr(char **ppa, char **ppb) {

	char *temp;

    temp = *ppa;      /* copy pa into temp */
    *ppa = *ppb;      /* copy pb into pa */
    *ppb = temp;      /* copy temp into pb */
}

int main()
{
    char *pa = "hello ";
    char *pb = "world!";

    printf("%s%s\n", pa, pb);
    swapstr(&pa, &pb);
    printf("%s%s\n", pa, pb);
    return 0;
}
\end{lstlisting}
\end{figure}

De uitvoer van dit programma is te zien in de figuur~\ref{fig:poiwisselen}. Te zien is dat de in de eerste regels de string op originele volgorde worden afgedrukt en in de twee regel in omgekeerde volgorde. Deze manier van herschikken van array's van strings is veel efficiënter dan strings kopiëren.

\begin{dosbox}[title=Verwisselen van twee stings.,label=fig:poiwisselen]
hello world!
world!hello
\end{dosbox}


\section{Array van pointers}
\index{pointer!array van}\index{array!van pointers}
Uiteraard kunnen we ook een array van pointers maken. We demonstreren dat aan de hand van een array van pointers naar karakters. Omdat het pointers zijn, kan een pointer ook wijzen naar het begin van een array van karakters, oftewel strings. Dit is te zien in figuur~\ref{fig:poiarrayofpointers}. Merk op dat de vier pointers naar karakters wijzen. Dat daar toevallig vier strings aan gekoppeld zijn, is vanuit het perspectief van de pointers niet belangrijk.

\begin{figure}[!ht]
\centering
\begin{tikzpicture}[pointerstyle]
\foreach \ii/\x/\y/\stri in {0/3/0.75/{z,o,n,d,a,g,\textbackslash 0},1/2.5/-0.5/{m,a,a,n,d,a,g,\textbackslash 0},2/4/-1.7/{d,i,n,s,d,a,g,\textbackslash 0},3/2/-3/{w,o,e,n,s,d,a,g,\textbackslash 0}} {
	\node [memlocarray] (nod\ii) at (0,-1*\ii*\unitsize) {};
    \node [memlocarray] (var\ii) at (\x,\y) {};
    \foreach \iii [count=\i from 0] in \stri {
        \node [memlocarray] at (\x+\i*\unitsize,\y) {\iii};
    }
    \draw[center*-latex] (nod\ii.center) -- (var\ii.west);
}
\draw (nod0) node [yshift=\unitsize cm] {day[]};

\end{tikzpicture}
\caption{Voorstelling van een array van poiinters naar strings.}
\label{fig:poiarrayofpointers}
\end{figure}

We declareren een array van vier pointers naar karakters. Daarna laten we de pointers naar strings wijzen. Zie listing~\ref{cod:poiarrayofpointers}.

\begin{figure}[!ht]
\begin{lstlisting}[caption=Een array van pointers.,label=cod:poiarrayofpointers]
char *day[4];          /* array of four pointers to char */

day[0] = "zondag";     /* points to the 'z' */
day[1] = "maandag";    /* points to the 'm' */
day[2] = "dinsdag";    /* points to the 'd' */
day[3] = "woensdag";   /* points to the 'w' */
\end{lstlisting}
\end{figure}

Omdat dit soort toekenningen veel voorkomen, mogen de strings ook bij declaratie aan de pointers worden toegekend. Dit is te zien in listing~\ref{cod:poiarrayofpointers2}.

\begin{figure}[!ht]
\begin{lstlisting}[caption=Een array van pointers naar strings met initialisatie.,label=cod:poiarrayofpointers2]
char *day[4] = {"zondag", "maandag", "dinsdag", "woensdag"};
\end{lstlisting}
\end{figure}

Opmerking: de C++-standaard verbiedt het gebruik van dit soort declaraties en initialisaties. Veel compilers geven echter een waarschuwing en gaan gewoon verder. Omdat de strings in dit soort constructies meestal niet veranderen, kan het keyword \texttt{const} voor de declaratie gezet worden, waarmee wordt aangegeven dat de strings niet veranderen. C++-compliers accepteren deze constructie.


\section{Pointers naar een array van pointers}
\index{pointer!naar array van pointers}
\label{sec:pointersnaareenarrayvanpointers}

We bekijken nu een wat complexer voorbeeld van het gebruik van pointers. We declareren een array \texttt{p} van vier pointers naar integers. We vullen de array met de adressen van de integers \texttt{i}, \texttt{j}, \texttt{k} en \texttt{l}. Daarna declareren we een pointer \texttt{pp} die wijst naar (het eerste element van) de array. Een voorstelling van de variabelen is te zien in figuur~\ref{fig:poipointertoarrayofpointers}.

\begin{figure}[!ht]
\centering
\begin{tikzpicture}[pointerstyle]
\foreach \ii/\x/\y/\name/\val in {0/3/0.75/i/3,1/2.5/-0.85/j/5,2/4/-2/k/2,3/2/-3/l/4} {
	\node [memlocarray] (nod\ii) at (0,-1*\ii*\unitsize) {};
    \node [memloc] (var\ii) at (\x,\y) {\val};
    \draw (var\ii) node [yshift=0.9*\unitsize cm] {\name};
    \draw[center*-latex] (nod\ii.center) -- (var\ii.west);
}
\draw (nod0) node [yshift=\unitsize cm] {p[]};
\node (AA) at (-3,0) [memloc] {};
\draw (AA) node [yshift=\unitsize cm] {pp};
\draw[center*-latex] (AA.center) -- (nod0.west);
\end{tikzpicture}
\caption{Voorstelling van een pointer naar een array van pointers naar integers.}
\label{fig:poipointertoarrayofpointers}
\end{figure}

De array \texttt{p} wordt gedeclareerd als:

\hspace*{1em}\texttt{int *p[4];}

De rechte haken hebben een hogere prioriteit dan de dereferentie-operator. We lezen de declaratie dus als: \texttt{p} is een array van vier elementen en elk element is een een pointer die wijst naar een integer. De pointer \texttt{pp} wordt gedeclareerd als:

\hspace*{1em}\texttt{int **pp;}

Let goed op wat hier staat: \texttt{pp} is een pointer naar een pointer naar een integer. Vanuit pointer \texttt{pp} is niet af te leiden dat \texttt{pp} wijst naar een array, alleen maar dat er twee dereferenties nodig zijn om bij een integer te komen. We moeten dat in het C-programma zelf scherp in de gaten houden.

We gebruiken de pointers zoals te zien is in listing~\ref{cod:poipointertoarrayofpointers}. in regel 6 worden de vier integers gedeclareerd. In regel 7 declareren we array \texttt{p} en initialiseren de array met de adressen van de integers. We geven geen array-grootte op want de C-compiler kan dat zelf uitrekenen.

\begin{figure}[!ht]
\begin{lstlisting}[caption=Voorbeeld van het gebruik van pointers.,label=cod:poipointertoarrayofpointers]
#include <stdio.h>
#include <stdlib.h>

int main(void) {

    int i=3, j=5, k=2, l=4;
    int *p[] = {&i, &j, &k, &l};
    int **pp = p;

    printf("&p: %p, &pp: %p, &p[0]: %p, &p[1]: %p, &p[2]: %p, &p[3]: %p\n\n", &p, &pp, &p[0], &p[1], &p[2], &p[3]);

    printf("&i: %p, p[0]: %p\n", &i, p[0]);
    printf("&j: %p, p[1]: %p\n", &j, p[1]);
    printf("&k: %p, p[2]: %p\n", &k, p[2]);
    printf("&l: %p, p[3]: %p\n", &l, p[3]);

    printf("\ni: %d, *p[0]: %d, **pp: %d\n", i, *p[0], **pp);

    return 0;
}
\end{lstlisting}
\end{figure}

In regel 8 declareren we de pointer \texttt{pp} (let op het gebruik van de dubbele dereferentie-operator) en initialiseren \texttt{pp} met het adres van \texttt{p}. In regel 10 drukken alle adressen van de pointers af. In de regels 12 t/m 15 drukken we de adressen van de integers en de inhouden van de array-elementen af. Om variabele \texttt{i} af te drukken hebben we drie mogelijkheden: \texttt{i} (de variabele), \texttt{*p[0]} (waar \texttt{p[0]} heen wijst) en \texttt{**pp}. Die wijst dus via dubbele dereferentie ook naar \texttt{i}.

Een mogelijke uitvoer is te zien in de figuur~\ref{fig:poienkelepointers}. We zeggen hierbij mogelijk omdat het draaien van het programma op een computer andere waarden (adressen) kan opleveren. In de onderstaande figuur is te zien dat pointer met acht hexadecimale cijfers worden afgedrukt.  Dat betekent dat de pointers met 32 bits worden opgeslagen. We hebben gebruik gemaakt van een 32-bits C-compiler.

\begin{dosbox}[title=Uitvoer van enkele pointers.,label=fig:poienkelepointers]
&p: 0061FDF0, &pp: 0061FDE8, &p[0]: 0061FDF0, &p[1]: 0061FDF8,
&p[2]: 0061FE00, &p[3]: 0061FE08

&i: 0061FE1C, p[0]: 0061FE1C
&j: 0061FE18, p[1]: 0061FE18
&k: 0061FE14, p[2]: 0061FE14
&l: 0061FE10, p[3]: 0061FE10

i: 3, *p[0]: 3, **pp: 3
\end{dosbox}

%Er zijn nog wel meer subtiele verschillen tussen pointers en array's mogelijk. Zo is \texttt{int *pa[13]} een array van 13 pointers naar ints.


\section{Argumenten meegeven aan een C-programma}
\index{argumenten!aan een C-programma}\index{command line argumenten}
\label{sec:argumentenmeegevenaaneencprogramma}
In besturingssystemen die C ondersteunen zoals Windows, Linux en Mac OS-X, is het mogelijk om een C-programma \textsl{command line argumenten} mee te geven. Bij het starten van een programma kunnen we (optioneel) gegevens invoeren en overdragen aan een gecompileerd C-programma. Als voorbeeld starten we het programma \texttt{myprog.exe} met de argumenten \texttt{argument1} en \texttt{argument2}. Het programma drukt eenvoudigweg alle argumenten op het beeldscherm af. Te zien is dat ook de programmanaam als argument wordt meegegeven.

\begin{dosbox}[title=Afdrukken van programmanaam en argumenten.,label=fig:poiargcv]
C:\Users\C> myprog.exe argument1 argument2

Aantal argumenten: 3

Argument 0: myprog.exe
Argument 1: argument1
Argument 2: argument2

C:\Users\C>
\end{dosbox}

Elk C-programma krijgt per definitie de twee parameters \texttt{argc} en \texttt{argv} mee, die aan \texttt{main} worden meegegeven. Dit is te zien in listing~\ref{cod:poiargs}.

\begin{figure}[!ht]
\begin{lstlisting}[caption=Declaratie van de command line parameters.,label=cod:poiargs]
int main(int argc, char *argv[]) {

    /* rest of the code */
    
    return 0;
}
\end{lstlisting}
\end{figure}

De integer \texttt{argc} (\textbf{arg}ument \textbf{c}ount) geeft aan hoeveel parameters aan het C-programma zijn meegegeven. De pointer \texttt{argv} (\textbf{arg}ument \textbf{v}ector) is  een pointer naar een lijst van pointers naar strings, gedeclareerd als \texttt{*argv[]}. Elke string bevat een argument. Per definitie wijst \texttt{argv[0]} naar een string waarin de programmanaam vermeld staat. Dat houdt in dat \texttt{argc} dus minstens~1 is. Er zijn dan geen optionele argumenten meegegeven.

In het voorbeeldprogramma is \texttt{argc} dus 3 en zijn \texttt{argv[0]}, \texttt{argv[1]} en \texttt{argv[2]} pointers naar respectievelijk \texttt{myprog.exe}, \texttt{argument1} en \texttt{argument2}. In figuur~\ref{fig:poiargcargv} is een uitbeelding van de variabelen \texttt{argc} en \texttt{argv} te zien. De strings worden, zoals gebruikelijk in C,  afgesloten met een nul-karakter. De C-standaard schrijft voor dat de lijst van pointers naar strings wordt afgesloten met een NULL-pointer.

\begin{figure}[!ht]
\centering
\begin{tikzpicture}[pointerstyle]
\node (argc) at (0,0) [memloc] {3};
\draw (argc) node [yshift=\unitsize cm] {argc};
\node (argv) at (3,0) [memloc] {};
\draw (argv) node [yshift=\unitsize cm] {argv};
\node (argv0) at (6,0) [memlocarray] {};
\draw (argv0) node [yshift=\unitsize cm] {argv[]};
\node (argv1) at (6,-\unitsize) [memlocarray] {};
\node (argv2) at (6,-2*\unitsize) [memlocarray] {};
\node (argv3) at (6,-3*\unitsize) [memlocarraynull] {};
\draw[center*-latex] (argv.center) -- (argv0.west);
\node (argv0s) at (8,0) [memloc,anchor=west] {myprog.exe\textbackslash 0};
\node (argv1s) at (8,-\unitsize) [memloc,anchor=west] {argument1\textbackslash 0};
\node (argv2s) at (8,-2*\unitsize) [memloc,anchor=west] {argument2\textbackslash 0};
\draw[center*-latex] (argv0.center) -- (argv0s.west);
\draw[center*-latex] (argv1.center) -- (argv1s.west);
\draw[center*-latex] (argv2.center) -- (argv2s.west);
\end{tikzpicture}
\caption{Voorstelling van de variabelen \texttt{argc} en \texttt{argv}.}
\label{fig:poiargcargv}
\end{figure}

Het programma \texttt{myprog.exe} is te zien in listing~\ref{cod:myprogexe}. Het
programma drukt eerst de variabele \texttt{argc} af. Met behulp van een \texttt{for}-lus
worden de argumenten \'e\'en voor \'e\'en afgedrukt. Merk op dat \texttt{argv[i]} een
pointer is naar het $i^e$ argument. We kunnen \texttt{argv[i]} dus direct gebruiken
voor het afdrukken van de bijbehorende string.

\begin{figure}[!ht]
\begin{lstlisting}[caption=Het programma myprog.exe.,label=cod:myprogexe]
#include <stdio.h>
#include <stdlib.h>

int main(int argc, char *argv[]) {

    int i;

    printf("\nAantal argumenten: %d\n\n", argc);
    for (i = 0; i<argc; i++) {
        printf("Argument %d: %s\n", i, argv[i]);
    }
    return 0;
}
\end{lstlisting}
\end{figure}

Merk op dat \texttt{argv} een echte pointers is en niet de naam van een array. We mogen \texttt{argv} dus aanpassen. Dit is te zien in listing~\ref{cod:poiargc2}.

\begin{figure}[!ht]
\begin{lstlisting}[caption=Afdrukken van argumenten.,label=cod:poiargc2]
#include <stdio.h>
#include <stdlib.h>

int main(int argc, char **argv) {

    printf("\nArguments: ");
    while (argc>0) {
        printf("%s ", *argv);
        argv = argv + 1;
        argc = argc - 1;
    }
    return 0;
}
\end{lstlisting}
\end{figure}


\section{Pointers naar functies}
\index{pointer!naar functie}
\label{sec:pointersnaarfunctie}
Functies zijn stukken programma die ergens in het geheugen liggen opgeslagen. De naam van een functie is het adres van de eerste instructie van de functie. Het is dus mogelijk om de naam van een functie te gebruiken als een pointer. Dit worden \textsl{functie-pointers} genoemd.

In listing~\ref{cod:poipointertofunction} is te zien hoe een functie-pointer wordt gedeclareerd. In de eerste regel wordt een functie gedefinieerd, een zogenoemde prototype, die een \texttt{int} als parameter meekrijgt en een \texttt{int} teruggeeft. In de tweede regel wordt \texttt{pf} gedeclareerd als een pointer naar een functie die een \texttt{int} meekrijgt en een \texttt{int} teruggeeft.

\begin{figure}[!ht]
\begin{lstlisting}[caption=Een functie en een pointer naar een functie.,label=cod:poipointertofunction,escapeinside={}{}]
int func(int a);   /* func is a function returning an int */
int (*pf)(int a);  /* pf is a pointer to a function
                                         returning an int */

pf = func;             /* assign pf */
\end{lstlisting}
\end{figure}

Let op het gebruik van de haakjes. Die zijn nodig om prioriteiten vast te leggen. Zonder de haken staat er \texttt{int *pf(int a)} en dan is \texttt{pf} een functie die een \texttt{int} als parameters meekrijgt en een pointer naar een \texttt{int} teruggeeft. Zie listing~\ref{cod:poipointertofunction2}.

\begin{figure}[H]
\begin{lstlisting}[caption=Een functie en een pointer naar een functie.,label=cod:poipointertofunction2,escapeinside={}{}]
int (*pf)(int a);  /* pf: pointer to a function returning an int */
int *pf(int a);    /* pf: function returning a pointer to an int */
\end{lstlisting}
\end{figure}

In figuur~\ref{fig:poipointertofunction} is te zien hoe \texttt{pf} wijst naar de functie \texttt{func}. We kunnen nu \texttt{pf} gebruiken om de functie aan te roepen. In de figuur is een aantal instructies gezet.

\begin{figure}[!ht]
\centering
\begin{tikzpicture}[pointerstyle]
\foreach \ii [count=\i from 0] in {LDR,ADD,SUB,MUL,CMP,BLO,RET} {
	\node [memloc] (nod\i) at (0,-1*\i*\unitsize) {\ii};
}
\draw (nod0) node [yshift=\unitsize cm] {func};
\node (AA) at (-3,0) [memloc] {};
\draw (AA) node [yshift=\unitsize cm] {pf};
\draw[center*-latex] (AA.center) -- (nod0.west);
\end{tikzpicture}
\caption{Voorstelling van een pointer naar een functie.}
\label{fig:poipointertofunction}
\end{figure}

Het gebruik van functie-pointers komt niet zo vaak voor in C-programma's. Het is meer in gebruik bij het schrijven van besturingssystemen. We willen toch even twee functies de revue laten passeren waarbij functie-pointers gebruikt worden.

\subsubsection*{Quicksort}
\index{quicksort}
Quicksort is een sorteeralgoritme ontworpen door C.A.R. Hoare in 1962. Het is een van de efficientste sorteeralgoritmes voor algemeen gebruik. De exacte werking zullen we niet bespreken, er zijn genoeg boeken die dit beschrijven. De standard library bevat een implementatie onder de naam \texttt{qsort}. Het prototype van \texttt{qsort} is:


\hspace*{1em}\texttt{void qsort(void *base, size\_t nitems, size\_t size,}\\
\hspace*{1em}\texttt{\ \ \ \ \ \ \ \ \ \ \ \ \ \ \ \ \ \ \ \  int (*compar)(const void *, const void*));}

Hierin is \texttt{base} een pointer naar het eerste element van de array, \texttt{nitems} het aantal elementen in de array en \texttt{size} de grootte (in bytes) van één element. De functie \texttt{compar} behoeft wat speciale aandacht. Dit is een functie (door de programmeur zelf te schrijven) die twee pointers naar twee elementen in de array meekrijgt. De pointers zijn van het type \texttt{void~*} want we weten niet wat het datatypes van de elementen van de array zijn. De functie \textsl{moet} een getal geven kleiner dan 0 als het eerste argument kleiner is dan het tweede element, 0 als de twee elementen gelijk zijn en een getal groter dan 0 als het eerste element groter is dan het tweede element.

\begin{figure}[!ht]
\begin{lstlisting}[caption=Sorteren van een array met quicksort.,label=cod:poiexampleqsort]
#include <stdio.h>
#include <stdlib.h>

int values[] = { 88, 56, 100, 2, 25 };

int cmpint (const void * a, const void * b) {
    /* explicit type case to pointers to integers */
    return ( *(int*)a - *(int*)b );
}

int main () {
    int n;

    printf("De array voor sorteren: \n");
    for( n = 0 ; n < 5; n++ ) {
        printf("%d ", values[n]);
    }

    qsort(values, 5, sizeof(int), cmpint);

    printf("\nDe array na sorteren: \n");
    for( n = 0 ; n < 5; n++ ) {
        printf("%d ", values[n]);
    }

    return(0);
}
\end{lstlisting}
\end{figure}

In listing~\ref{cod:poiexampleqsort} is een programma te zien dat een array van integers sorteert. De functie \texttt{cmpint} vergelijkt twee integers uit de array. Let op de constructie om bij de integers te komen. De pointers zijn van het type \texttt{void~*} dus er is een expliciete type case nodig naar een pointer naar een integer. Dat wordt gerealiseerd door \texttt{(int~*)}. Daarna wordt de pointer gebruikt om bij de integer te komen. Om parameter \texttt{a} te verkrijgen is dus \texttt{*(int~*)~a} nodig. Op deze wijze wordt ook parameter \texttt{b} gevonden en de functie geeft eenvoudigweg het verschil tussen de twee parameters terug.

%\begin{figure}[!ht]
%\begin{lstlisting}[caption=Voorbeeld van het gebruik van functie-pointers.]
%#include <stdio.h>
%#include <stdlib.h>
%
%void taska(void) {
%    printf("Task A\n");
%}
%
%void taskb(void) {
%    printf("Task B\n");
%}
%
%void taskc(void) {
%    printf("Task C\n");
%}
%
%int main() {
%
%    /* Declare an array of function pointers and assign addresses */
%    void (*func_ptr[])(void) = {taska, taskb, taskc};
%
%    /* Calculate size of array */
%    int size = sizeof(func_ptr)/sizeof(func_ptr[0]);
%    int i;
%
%    /* Loop through all array elements */
%    for (i=0; i<size; i++) {
%        /* Start the function */
%        (*func_ptr[i])();
%    }
%
%    return 0;
%}
%\end{lstlisting}
%\end{figure}


\section{Dynamische geheugenallocatie}
\index{dynamische geheugenallocatie}
\index{geheugenallocatie}
Bij het schrijven van een programma declareren we de variabelen die we nodig hebben. Deze geheugenplaatsen blijven in principe bezet. Wel is het zo dat lokale variabelen worden aangemaakt en afgebroken aan het begin respectievelijk het einde van een functie (tenzij ze als \texttt{static} gekwalificeerd zijn), maar er komt geen extra geheugen erbij. Via het besturingssysteem is het mogelijk om extra geheugenruimte aan te vragen als dat nodig is. Is de aangevraagde geheugenruimte niet meer nodig dan geven we het terug aan het besturingssysteem. We noemen dit \textsl{dynamische geheugenallocatie}.

De standard library kent een aantal functie op dit gebied. Om deze functies te gebruiken moet het header-bestand \texttt{malloc.h}\indextwo{malloc.h}{header-bestand} geladen worden. De functie \texttt{malloc}\indextwo{malloc}{functie} vraagt aan het besturingssysteem extra geheugenruimte \textsl{in bytes}. Als dat lukt, dan wordt een void-pointer\index{void-pointer} naar het begin van de geheugenruimte teruggegeven. De void-pointer moet dus gecast worden naar het juiste type. Het geheugen wordt niet geïnitialiseerd, bijvoorbeeld met nul-bytes\index{nul-bytes}. Als het niet lukt, wordt een NULL-pointer\index{NULL-pointer} teruggegeven. Hier moet uiteraard op getest worden. De functie \texttt{free}\indextwo{free}{functie} geeft eerder gealloceerd geheugen weer vrij. Als argument wordt een pointer naar het gealloceerde geheugenruimte meegegeven. Let erop dat de geheugenruimte niet eerder is vrijgegeven, want dan stopt het programma met een crash.

Een van de meest voorkomende problemen is dat gealloceerd geheugen niet wordt vrijgegeven. Dit wordt een \textsl{memory leak}\index{memory leak} genoemd. Op zich is dat niet zo erg, want als het programma stopt, ruim het besturingssysteem alle gealloceerde geheugenruimtes op. Maar als een programma alsmaar geheugen vraagt en niets vrijgeeft, kan het zijn dat uiteindelijk geen geheugen meer beschikbaar is.

Een eenvoudig gebruik van \texttt{malloc} en \texttt{free} is te zien in listing~\ref{cod:mem_alloc}. Regel 6 zorgt ervoor dat Visual Studio de functie \texttt{strcpy}\indextwo{strcpy}{functie} accepteert. In regel 8 declareren we de constante \texttt{SIZE} met de waarde van 1000. De constante is van het type \texttt{size\_t}. In regel 17 declareren we een karakterpointer en initialiseren we die met \texttt{NULL}. Regel 19 roept de functie \texttt{malloc} aan om 1000 bytes te alloceren. De functie geeft een void-pointer terug, dus we moeten een type cast naar \texttt{char *} doen. We testen in de regels 16 t/m 19 of de allocatie gelukt is en zo niet, dan drukken we een foutmelding af en stoppen we het programma.

\booklisting[]{C}{Gebruik van \texttt{malloc} en \texttt{free}}{mem_alloc}{c}{!ht}

In regel 21 t/m 23 kopiëren we een string naar het geheugen en drukken het af. In regel~25 geven we het geheugen weer vrij. De lezer wordt gevraagd om het programma te testen met verschillende waarden voor \texttt{SIZE}.

We vermelden nog even twee andere functies: \texttt{realloc} en \texttt{calloc}. De functie \texttt{realloc}\indextwo{realloc}{functie} heralloceert een stuk geheugen. Met de aanroep

\hspace*{1em}\texttt{pnew = realloc(porig, size);}

wordt het stuk geheugen waar \texttt{porig} naar wijst uitgebreid of ingekrompen, \texttt{pnew} wijst naar de nieuwe geheugenruimte. Er zijn een paar zaken waar we op moeten letten: 1) als het geheugen wordt ingekrompen, blijft de originele inhoud ongewijzigd; 2) als het geheugen wordt uitgebreid blijft de originele inhoud ongewijzigd en het nieuwe, extra geheugen wordt niet geïnitialiseerd, 3) als het niet lukt om het geheugen op de originele plaats uit te breiden, wordt een compleet nieuw geheugenblok gealloceerd en wordt de inhoud van het originele inhoud gekopieerd, de rest wordt niet geïnitialiseerd, het oude geheugenblok wordt vrijgegeven; 4) als er geen geheugenruimte meer vrij is wordt een NULL-pointer teruggegeven en blijven de originele pointer en geheugen intact.

De functie \texttt{calloc}\indextwo{calloc}{functie} alloceert een aantal veelvouden van een elementair datatype en vult de geheugenruimte met nul-bytes\index{nul-bytes}. Dus

\hspace*{1em}\texttt{int *pint = calloc(1000, sizeof(int));}

alloceert 1000 integers en vult ze met nul-bytes. Als het niet lukt, wordt een NULL-pointer teruggegeven.


\advanced
\section{Pointers naar vaste adressen}
\index{pointer!naar vast adres}
C is erg coulant bij het toekennen van adressen aan pointers. Zo is het mogelijk om een pointer naar een vast adres te laten wijzen.
De toekenning en initialisatie

\hspace*{1em}\texttt{unsigned int *p = (unsigned int *) 0x40fe;}

zorgt ervoor dat \texttt{p} een pointer is naar een unsigned integer op adres 40FE$_{16}$ (hexadecimale notatie). Er is een expliciete type cast nodig om de integer \texttt{0x40fe} om te zetten naar een adres. We kunnen nu iets in die geheugenplaats zetten door een dereferentie:

\hspace*{1em}\texttt{*p = 0x3ff;}


Nu zullen dit soort toekenningen niet voorkomen op systemen waar een besturingssysteem op draait. Het gebruik van de pointer zal hoogst waarschijnlijk een crash van het programma veroorzaken. Maar op kleine computersystemen zonder besturingssysteem, de zogenoemde \textsl{bare metal}-systemen\index{bare metal system}, is het vaak de enige manier om informatie naar binnen en naar buiten te krijgen.

We geven in listing~\ref{cod:poiatmega} een voorbeeld van het gebruik van dit soort pointers op een ATmega32-microcontroller van Atmel. Overigens bevat de toolchain\index{toolchain} header-bestanden waarin deze definities al staan.

%%%We kunnen ook gebruik maken van een ``tijdelijke'' pointer:
%%%
%%%\hspace*{1em}\texttt{\#define DDRB (*(volatile uint8\_t*)0x37)}
%%%
%%%We kunnen nu \texttt{DDRB} gebruiken alsof het een gewone C-variabele is.

\begin{figure}[!p]
\begin{lstlisting}[caption=Het gebruik van pointers op een ATmega-microcontroller.,label=cod:poiatmega]
typedef unsigned char uint8_t;

int main(void) {

    uint8_t c1, c2, i;
    volatile int i;

    // Pointers to addresses
    volatile uint8_t* ddrb = (uint8_t*) 0x37;   // DDRB address
    volatile uint8_t* portb = (uint8_t*) 0x38;  // PORTB address

    *ddrb = 0xFF; // Set PORTB to all outputs

    while (1) {
        c1 = 0x80;
        c2 = 0x01;

        for (i = 0; i < 4; i++) {
            for (i = 0; i < 30000; ++i); // wait a bit
            *portb = ~(c1 | c2);  // Leds on Port B are active low
            c1 >>= 1;
            c2 <<= 1;
        }
    }

    return 0;
}
\end{lstlisting}
\end{figure}


\section{Subtiele verschillen en complexe declaraties}
We zullen in de praktijk nauwelijks complexere situaties tegenkomen dan dat we tot nu toe zijn tegengekomen. Toch willen we een bloemlezing geven van enkele bekende en onbekende, complexe declaraties.

%% Bereken de breedte van de tweede colom
%\newdimen\delengte\delengte=\textwidth\advance\delengte by -4.6cm\advance\delengte by -2\tabcolsep
% Bereken de breedte van de eerste kolom
\setbox0\hbox{\texttt{char (*(*x[3])())[5]}}
\newdimen\delengte\delengte=\textwidth\advance\delengte by -\wd0\advance\delengte by -2\tabcolsep
\begin{table}[!p]
\begin{tabular}{@{}lp{\delengte}@{}}
\texttt{int *p} & \texttt{p}: pointer to int\\
\texttt{int **pp} & \texttt{pp}: pointer to pointer to int \\
\texttt{int ***ppp} & \texttt{ppp}: pointer to pointer to pointer to int \\
\texttt{int **pp[3]} & \texttt{pp}: array[3] of pointer to pointer to int \\
\texttt{char **argv} & \texttt{argv}: pointer to pointer to char \\
\texttt{char *argv[]} &\texttt{argv}: array[] of pointer to char \\
\texttt{int *list[5]} & \texttt{list}: array[5] of pointer to int \\
\texttt{int (*list)[5]} & \texttt{list}: pointer to array[5] of int \\
\texttt{int *(*list)[5]} & \texttt{list}: pointer to array[5] of pointer to int \\
\texttt{int *pf()} & \texttt{pf}: function returning a pointer to int \\
\texttt{int (*pf)()} & \texttt{pf}: pointer to function returning an int \\
\texttt{int *(*pf)()} & \texttt{pf}: pointer to function returning a pointer to int \\
\texttt{char (*(*x())[])()} & \texttt{x}: function returning pointer to array[] of pointer to function returning char.\\
\texttt{char (*(*x[3])())[5]} & \texttt{x}: array[3] of pointer to function returning pointer to array[5] of char.\\
\end{tabular}
\end{table}
\basic
