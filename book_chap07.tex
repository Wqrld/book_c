\chapter{Structures}
\label{cha:structures}
\thispagestyle{empty}

Een structuur (Engels: structure\footnote{We zullen vanaf nu de Engelse naam structure gebruiken.}) is een verzameling bij elkaar behorende gegevens beschikbaar onder één enkele naam. Dit is vooral handig bij complexe \textsl{datastructuren} want ze helpen om gerelateerde variabelen als een eenheid te bewerken (denk hierbij aan argumenten en parameters van een functie) in plaats van een aantal verschillende variabelen. Structures komen overeen met \textsl{rijen} in een relationele database. Een bekend voorbeeld is de loonlijst van werknemers. Een werknemer heeft een naam, een adres, een salaris en een functie.

Een structure kan gezien worden als één enkele variabele. Zo mogen we structures eenvoudig kopiëren met een toekenning, we mogen ze meegeven als argumenten aan een functie, een functie kan een structure in zijn geheel teruggeven en we mogen de variabelen binnen de structure gebruiken in expressie. We mogen \textsl{niet} twee structures vergelijken.


\section{Definitie van structures}
Een welhaast klassiek voorbeeld van een structure is een aantal artikelen. Van de artikelen geven we het artikelnummer, de naam, het aantal dat beschikbaar is en de prijs per stuk op. We declareren de structure \texttt{artikel} als volgt (listing~\ref{cod:struct1}).

% We gebruiken label=cod:struct1 om de standaard label te wijzigen
\booklistingfromproject[linerange={6-11},label=cod:struct1]{C}{Declaratie van de structure \texttt{artikel}}{struct_artikel}{c}{!ht}

Hierin wordt \texttt{artikel} de \textsl{structure identier}\index{structure identifier} genoemd. De variabelen binnen de structure worden \textsl{leden} genoemd maar we zullen gebruik maken van de Engelse naam \textsl{members}\index{member}\index{structure!member}\indexkeyword{struct}.

We kunnen nu de structure gebruiken in definities door gebruik te maken van

\hspace*{1em}\texttt{struct artikel} \textsl{variabele}\texttt{;}

om variabelen te definiëren. In listing~\ref{cod:struct2} is een aantal definities te zien:

\booklistingfromproject[linerange={40-43},label=cod:struct2]{C}{Definitie van enkele variabelen}{struct_artikel}{c}{!ht}

In regel 1 wordt de variabele \texttt{floppy} gedefinieerd zonder initialisatie. We kunnen bij definitie gelijk de members initialiseren zoals te zien is in regels 3 en 4. In de initialisatielijst moeten natuurlijk wel de juiste datatypes gebruikt worden, tenzij automatische conversie mogelijk is. Structures mogen zowel globaal als lokaal worden gedefinieerd.

\section{Toegang tot members}
Om gebruik te maken van een member gebruiken we de \textsl{member operator} \texttt{.}\indexop{.} (punt). Om de prijs van het artikel \texttt{floppy} in te stellen kunnen we bijvoorbeeld gebruiken:

\hspace*{1em}\texttt{floppy.prijs = 10.73;}

Om het aantal af te drukken gebruiken we \texttt{printf} met de member \texttt{aantal}:

\hspace*{1em}\texttt{printf("Aantal beschikbaar is \%d", floppy.aantal);}

Om het aantal beschikbare exemplaren van een artikel te verhogen kunnen we de member \texttt{aantal} verhogen:

\hspace*{1em}\texttt{floppy.aantal += 5; \ \ // 5 exemplaren erbij}

Let erop dat de naam van een artikel een \textsl{string} is. Om de naam aan te passen moeten we gebruik maken van de functie \texttt{strcpy}\indexfunc{strcpy}:

\hspace*{1em}\texttt{strcpy(floppy.naam, "Floppy 3.5 in");}


\section{Toekennen van structures}
Een structure mag toegekend worden aan een andere structure middels de \texttt{=}-operator. We mogen dus schrijven

\begin{lstlisting}[style=lstoneline]
struct artikel a, b;
...
a = b;
\end{lstlisting}

Hierbij worden de members van structure \texttt{b} naar de overeenkomstige members van structure \texttt{a} gekopieerd. In feite is \texttt{a} nu een kopie van \texttt{b}.


\section{Vergelijken van structures}
Een structure mag niet zomaar met een andere structure vergeleken worden, ook niet als de structures van hetzelfde type zijn. Dus

\begin{lstlisting}[style=lstoneline]
struct artikel a, b;
..
if (a == b) {        /* WRONG */
    ...
}
\end{lstlisting}

genereert een foutmelding door de compiler. De juiste manier is door alle members van de structures een voor een (\textsl{member-wise})\index{member-wise} te vergelijken. Dus

\begin{lstlisting}[caption=Vergelijken van twee structures.]
struct artikel a, b;
..
if (a.nummer == b.nummer && strcmp(a.naam, b.naam) == 0 &&
    a.aantal == b.aantal && a.prijs == b.prijs) {
    /* true */
} else {
    /* false */
}
\end{lstlisting}






\section{Functies met structures}
Stel dat we de gegevens van een artikel willen afdrukken. Dan kunnen we natuurlijk alle members apart afdrukken. Maar is het handiger om een functie te definiëren die dat voor ons doet. Een structure mag gewoon als argument aan een functie worden weergegeven\index{structure!als argument} of dienen als parameter in een functie\index{structure!als parameter}. In de functie kunnen we dan de members een voor een afdrukken. Dit is te zien in listing~\ref{cod:struct3}. We hebben extra spaties toegevoegd om de gegevens van elkaar te scheiden.

\booklistingfromproject[linerange={13-18},label=cod:struct3]{C}{Afdrukken van de gegevens van een  artikel}{struct_artikel}{c}{!ht}

Als we een nieuw artikel willen toevoegen, kunnen we alle members van de structure een waarde toekennen. Maar het is gemakkelijker om een functie te definiëren die dat voor ons doet. Een mogelijke functie is te zien in listing~\ref{cod:struct4}.

\booklistingfromproject[linerange={20-30},label=cod:struct4]{C}{Aanmaken van een nieuw artikel}{struct_artikel}{c}{!ht}

De parameters krijgen de waarden mee die aan de members moeten worden toegekend. Om een nieuwe structure te maken waar de gegevens inkomen, definiëren we in regel 3 een structure met de naam \texttt{nieuw}. We moeten daarna een voor een de waarden aan de members toekennen. Vervolgens geven we de gehele structure terug aan de aanroeper.

We kunnen de structure \texttt{floppy} bijvoorbeeld initialiseren met:

\hspace*{1em}\texttt{floppy = maak\_artikel(7, "Floppy", 5, 10.73);}


\begin{infobox}[Relationele databases]
Het gebruik van structures is nauw gekoppeld aan \textsl{relationele databases}. We kunnen voor onze winkel bijvoorbeeld \textsl{tabellen} aanmaken met klantgegevens, artikelen en bestellingen. Bij de klantgegevens voeren we de uniek klantnummer, naam, het adres en telefoonnummer in, bij de artikelen een uniek artikelnummer, de naam, het aantal dat beschikbaar is en de prijs per stuk. Als we een bestelling invoeren, hoeven we naast een uniek bestelnummer alleen het nummer van de klant in te voeren; we kunnen de klantgegevens via de klanttabel vinden. Als het adres van de klant verandert, dan hoeven we alleen de klanttabel te veranderen. Natuurlijk moet een lijst met gekochte artikelen worden ingevoerd, maar we hoeven dan alleen het artikelnummer en het aantal in te voeren, de rest is via de artikeltabel te vinden. Het uit elkaar trekken van al die gegevens wordt \textsl{normalisatie} genoemd en zorgt ervoor dat spaarzaam met opslag wordt omgegaan en dat gegevens niet meerdere keren in de database voorkomen (redundantie).
\end{infobox}



\section{Typedef}
\index{structure!typedef}
Met behulp van het keyword \texttt{typedef}\indexkeyword{typedef} kunnen we een structure beschikbaar stellen onder een eigen datatype. Zo kunnen we het nieuwe type \texttt{artikel\_t} aanmaken met de definitie in listing~\ref{cod:struct11}. Let erop dat \texttt{artikel\_t} geen echt nieuw datatype is, het is meer een synoniem. Het blijft onder alle omstandigheden een structure. We mogen dan ook de naam na \texttt{struct} achterwege laten. Merk op dat de definitie van de oorspronkelijke structure tussen \texttt{typedef} en \texttt{artikel\_t} staat.

\booklistingfromproject[linerange={6-11},label=cod:struct11]{C}{De definitie van datatype \texttt{artikel\_t}}{struct_artikel_typedef}{c}{!ht}

Vervolgens kunnen we variabelen definiëren met de \texttt{typedef}:

\hspace*{1em}\texttt{artikel\_t floppy, sdcard, usbstick;}

We kunnen \texttt{typedef} ook gebruiken bij andere datatypes:

\hspace*{1em}\texttt{typedef unsigned long int ulint;}

\texttt{Typedef}'s zijn handig om een programma onafhankelijke te maken van de C-compiler die gebruikt wordt en de computer waarop het programma draait. Als een programma verplaatst wordt naar een andere computer (dat wordt \textsl{porteren} genoemd), hoeven alleen de \texttt{typedef}'s te worden aangepast. Daarna kan het programma gecompileerd worden met de nieuwe \texttt{typedef}'s.


\section{Array van structures}
\index{array!van structures}\index{structure!array van}
Natuurlijk is het niet handig om voor elk artikel een variabele te definiëren. We kunnen dan veel beter gebruik maken van een array. De definitie

\hspace*{1em}\texttt{artikel\_t art[10];}

zorgt ervoor dat we een array van tien artikelen kunnen gebruiken. Om een element uit de array te selecteren gebruiken we de blokhaken \texttt{[]}. Die hebben een lagere prioriteit dan de member operator \texttt{.} (punt) dus er zijn geen haakjes nodig om een member te gebruiken:

\hspace*{1em}\texttt{art[6].aantal += 3;}

zorgt ervoor dat \texttt{aantal} van \texttt{art[6]} met 3 wordt verhoogd. Bij het meegeven van de array aan een functie moeten we expliciet het aantal arrayelementen opgeven omdat een array middels een adres naar het eerste element wordt meegegeven. Zie ook hoofdstuk~\ref{sec:pointersalsfunctieargumenten}.

In listing~\ref{cod:struct14} is een functie te zien die alle gegevens van de artikelen afdrukt. We gaan hier ervan uit dat als een artikel het nummer 0 heeft, dit artikel niet is ingevuld en drukken we de gegevens niet af. 

\booklistingfromproject[linerange={29-36},label=cod:struct14,escapeinside={{}{}}]{C}{Een functie om alle gegevens van artikelen af te drukken}{struct_artikel_typedef}{c}{!ht}

Het aantal elementen kunnen we laten uitrekenen met behulp van de operator \texttt{sizeof}\indexop{sizeof}. We delen de grootte van de totale array door de grootte van één element om het aantal elementen te berekenen.
We roepen de functie aan met

\begin{lstlisting}[style=lstoneline]
int aantal = sizeof art / sizeof art[0];
print_artikelen(art, aantal);
\end{lstlisting}
%\hspace*{1em}\texttt{print\_artikelen(art, sizeof art / sizeof art[0]);}


\section{Structures binnen structures}
\index{structure!binnen structure}
We kunnen ons programma uitbreiden met bestellingen. We definiëren daartoe drie structures voor artikelen, klanten en bestellingen. De structures zijn te zien in listing~\ref{cod:bestel1}. We merken op dat de structure \texttt{bestelling\_t} naast een uniek bestelnummer ook variabelen hebben van de structure \texttt{klant\_t} en van een array van \texttt{artikel\_t} (we gaan er gemakshalve vanuit dat een bestelling niet meer dan tien verschillende artikelen kan bevatten). We hebben dus structures binnen een structure. Als we een bestelling willen aanmaken dan moeten we naast het bestelnummer ook de klantgegevens en de artikelen opgeven.

\booklistingfromproject[linerange={6-23},label=cod:bestel1,escapeinside={{}{}}]{C}{De structures die horen bij een bestelling}{struct_bestelling}{c}{!ht}

Om een bestelling aan te maken definiëren we eerst een variabele van het nieuwe datatype \texttt{bestelling\_t} en initialiseren we de structure met nullen. Dit is te zien in listing~\ref{cod:bestel2}. Dat kan door alleen het bestelnummer expliciet met een 0 te initialiseren, de andere members wordt automatisch op 0 gezet. Daarna vullen we het bestelnummer in via

\hspace*{1em}\texttt{bestel.bestelnr = 123;}

Om het klantnummer op te geven moeten we via de structure \texttt{klant} in \texttt{bestelling} een waarde opgeven:

\hspace*{1em}\texttt{bestel.klant.klantnr = 73}

We kunnen een artikel bij de bestelling voegen door een artikel in zijn geheel te kopiëren naar een element van de variabele \texttt{artbestel}

\hspace*{1em}\texttt{bestel.artbestel[0] = art[2];}

We moeten dan alleen nog het aantal en de prijs berekenen. Dit is te zien in listing~\ref{cod:bestel2}.

\booklistingfromproject[linerange={46-53},label=cod:bestel2,escapeinside={{}{}}]{C}{Het vormgeven van een bestelling}{struct_bestelling}{c}{!ht}

Overigens is het niet slim om alle informatie van een artikel in de structure \texttt{bestel} op te nemen. Als de naam van de klant verandert, dan moeten we in twee structures de naam aanpassen. We kunnen beter alleen het klantnummer opnemen. De gegevens van de klant zijn dan via de klantstructures op te vragen. Zie ook het kader onder aan de pagina.


\section{Teruggeven van meerdere variabelen}
In C kan een functie slechts één variabele teruggeven. Als we meerdere variabelen willen teruggeven, kunnen we gebruik maken van een structure. De structure `verpakt' dan de variabelen onder één noemer.

Een bekend voorbeeld is het berekenen van de oplossingen van een kwadratische vergelijking met de vorm $ax^2+bx+c=0$. We kunnen de \textsl{wortels} (getallen waarvoor de functie~0 oplevert) berekenen met de wortelformule\index{wortelformule}:
%
\begin{equation}
x_{1,2} = \dfrac{-b\pm\sqrt{b^2-4ac}}{2a}
\end{equation}
%
De expressie onder het wortelteken heet de \textsl{discriminant} en de wortel van de discriminant heeft alleen een reële waarde als deze groter is dan of gelijk is aan 0. Verder mag $a$ niet~0 zijn, want dan kunnen we niet delen. We moeten nu in feite drie gegevens bepalen: de wortels $x_1$ en $x_2$ en of de wortels geldig zijn. We pakken deze drie gegevens in in een structure. Het volledige programma is te zien in listing~\ref{cod:struct_wortels}.

\booklistingfromproject[escapeinside={{}{}}]{C}{Een programma om de wortels van een kwadratische vergelijking te berekenen}{struct_wortels}{c}{!p}

We definiëren een structure met de drie gegevens zoals te zien is in regels 4 t/m 8. In de functie \texttt{bereken\_wortels} berekenen we de wortels als dat mogelijk is en we vullen de status van de wortels in. Kunnen ze niet berekend worden dan zetten we variabele \texttt{geldig} op 0, anders zetten we die variabele op 1. Aan het einde van de functie geven we een nieuw aangemaakte variabele terug. In de functie \texttt{print\_wortels} drukken we de wortels af of dat de wortels niet geldig zijn (kunnen niet berekend worden).

Merk overigens op dat als de discriminant 0 is, de wortels $x_1$ en $x_2$ dezelfde waarde hebben.


\advanced
\section{Unions}
Stel dat we onze eigen C-compiler willen ontwerpen (begin er trouwens niet aan, dat is heel complex). Dan moeten we een lijst bijhouden van gedefinieerde variabelen. Van de variabele moet dan ook het type worden opgeslagen en mogelijk een initiële waarde\footnote{Natuurlijk kan de waarde van een variabele tijdens runtime veranderen, maar het zou kunnen dat de compiler erachter komt dat een variabele op een bepaald moment een bekende waarde heeft. Deze waarde kan dan gebruikt worden bij een expressie of initialisatie.}. We kunnen dan een structure definiëren die alle mogelijke datatypes bevat, maar dat is verkwisten van geheugenruimte; een variabele kan immers maar één datatype hebben. We kunnen dan gebruik maken van een \textsl{union}\indexkeyword{union}.

Binnen een union wordt ruimte gereserveerd voor het grootste datatype (in het aantal bits). In listing~\ref{cod:union1} is een union variabele gedefinieerd met de naam \texttt{value}.

\booklistingfromproject[gobble=4,linerange={11-15},label=cod:union1]{C}{Een union om verschillende datatypes te gebruiken}{union}{c}{!ht}

De gereserveerde geheugenruimte is gelijk aan de geheugenruimte van de grootste variabele, in dit geval een \texttt{double}. We kunnen de union opnemen in een structure met daarin de naam en het type (en natuurlijk de union). Een voorbeeld is te zien in listing~\ref{cod:union2}. We hebben tevens gebruik gemaakt van een \textsl{enumeratie}\index{enumeratie} (zie paragraaf~\ref{sec:enumeraties}) voor de verschillende datatypes.

\booklistingfromproject[linerange={6-16},label=cod:union2,escapeinside={{}{}}]{C}{Een structure om een variabele in een compiler te gebruiken}{union}{c}{!ht}

We kunnen nu een variabele definiëren en initialiseren met gegevens. Dit is te zien in listing~\ref{cod:union3}. Te zien is dat de variabele wordt ingesteld als \texttt{float} en dat de waarde 3,14 bedraagt.

\booklistingfromproject[linerange={40-40,43-46},label=cod:union3,escapeinside={{}{}}]{C}{Initialiseren van een variabele}{union}{c}{!ht}

We kunnen op elk moment het type en de waarde veranderen (dat zou in een C-compiler natuurlijk niet kunnen gebeuren). We kunnen de variabele bijvoorbeeld kenmerken als een integer en er een waarde aan toekennen.

\booklistingfromproject[linerange={48-49},label=cod:union4,escapeinside={{}{}}]{C}{Initialiseren van een variabele}{union}{c}{!ht}

Bij het afdrukken van de waarde van de variabele raadplegen we het type. Met behulp van een \texttt{switch}-statement selecteren we hoe de variabele moet worden afgedrukt. Zie listing~\ref{cod:union5}.

\booklistingfromproject[linerange={18-36},label=cod:union5,escapeinside={{}{}}]{C}{Een functie om een variabele af te drukken}{union}{c}{!ht}


\section{Bitvelden}
\index{bit fields}
Bitvelden (bit fields) maken het mogelijk om delen van een geheugenplaats (of meerdere geheugenplaatsen gegroepeerd als één geheel), onder te verdelen in afzonderlijke bits. Dit is vooral handig bij het aansturen van hardware.

Stel dat we een (fictieve) hardwarematige \textsl{teller} (Engels: counter) willen aansturen. De gegevens van de teller worden met 32 bits weergegeven. In listing~\ref{cod:bitfields} is de indeling van de~32 bits te zien. De tellerwaarde bestaat uit 16 bits, dus de teller kan tellen van 0 t/m 65535. Dit wordt weergegeven met het veld \texttt{value}. De teller telt \textsl{cyclisch} dus na de hoogste waarde volgt weer de laagste waarde. De teller heeft dan een \textsl{overflow} gehad. Dit wordt weergegeven met het veld \texttt{overflow}. De teller heeft acht mogelijke manieren van tellen (omhoog, omlaag etc.). Dit wordt weergegeven met het veld \texttt{mode}. Als een speciale \textsl{interruptfunctie} gestart moet worden, geven we dat aan met het veld \texttt{interrupt\_enable}.

\begin{figure}[!ht]
\begin{lstlisting}[caption=Voorbeeld van bitvelden.,label=cod:bitfields]
struct {
    unsigned int interrupt_enable : 1; /* Bit 31 */
    unsigned int padding : 0;          /* Bits 30 down to 21 */
    unsigned int mode : 3;             /* Bits 20 down to 18 */
    unsigned int enable : 1;           /* Bit 17 */
    unsigned int overflow : 1;         /* Bit 16 */
    unsigned int value : 16;           /* Bits 15 down to 0 */
} counter;
\end{lstlisting}
\end{figure}
Het veld \texttt{padding} heeft de speciale breedte 0 om ervoor te zorgen dat de velden worden \textsl{opgelijnd} naar de breedte die van nature door de computer worden verwerkt. Een veel gebruikte term hiervoor is \textsl{machinewoord}. De bit \texttt{interrupt\_enable} komt daardoor op positie 31 terecht, de andere bits lopen van 0 t/m 20.


Om te teller te starten gebruiken we:

\hspace*{1em}\texttt{counter.enable = 1;}

Om de tellerwaarde te kopiëren naar een variabele gebruiken we:

\hspace*{1em}\texttt{unsigned int value = counter.value;}

Om te testen of een overflow heeft plaatsgevonden, gebruiken we:

\hspace*{1em}\texttt{if (counter.overflow == 1) \{ ... \} }

Alles aan het gebruik van bitvelden is afhankelijk van de hardware waarop het programma draait en de C-compiler die gebruikt wordt. Dat betekent dat programma's met bitvelden moeilijk overdraagbaar zijn naar andere computers en andere C-compilers. Bij systemen die van nature met 32 bits werken worden de bits (hoogst waarschijnlijk) samengepakt in 32 bits eenheden.

Bitvelden werden vroeger gebruikt om geheugenruimte te besparen. Tegenwoordig is dat niet meer nodig omdat de meeste computers genoeg geheugen hebben. Het gebruik van bitvelden wordt daarom ook afgeraden.

\basic
