%%%%%%%%%%%%%%%%%%%%%%%%%%%%%%%%%%%%%%%%%%%%%%%%%%%%%%%%%%%%%%%%%%%%%%%%%%%%%%%
%%%
%%%   VOORWOORD
%%%
%%%%%%%%%%%%%%%%%%%%%%%%%%%%%%%%%%%%%%%%%%%%%%%%%%%%%%%%%%%%%%%%%%%%%%%%%%%%%


\chapter{Voorwoord}
\label{cha:voorwoord}
\thispagestyle{empty}

Een van de beste boeken over C is
\textsl{The C Programming Language} van Brian Kernighan en Dennis
Ritchie~\cite{kernighan1988c}. Het boek is kort en bondig maar laat toch
ruimte om de tekst te ondersteunen met voorbeelden. Helaas zijn veel van
de voorbeelden lastig te volgen voor de beginnende programmeur. Zoals de
schrijvers zelf opmerken:

\begin{displayquote}
The book is not an introductory programming manual; it assumes some familiarity
with basic programming concepts like variables, assignment statements,
loops, and functions.
[...]
C is not a big language, and it is not well served by a big book.
\end{displayquote}

De schrijvers geven aan dat hun boek eigenlijk niet bedoeld is om de taal te leren
en dat een boek over de beginselen van C helemaal niet zo groot hoeft te zijn.
Dat heeft ook te maken met hoe rijk de taal is. C is gebaseerd op een aantal
kenmerkende taalconstructies die ook bij vele andere programmeertalen voorkomen.
Daarom is dit boek bedoeld als inleiding op de programmeertaal C en is niet
toereikend om alle facetten van de taal te leren. Dat is ook niet de
bedoeling van dit boek. We laten alleen maar zien wat gebruikelijk is bij
het ontwikkelen van C-programma's. 

Veel boeken over C beschrijven het ontwikkelen van programma's op het Unix
operating system en afgeleide varianten zoals Linux, FreeBSD en Mac OS-X.
Dat is echter niet de werkwijze van veel programmeurs. Veel software wordt
ontwikkeld met een Integrated Development Environment (IDE). Bekende IDE's
zijn Visual Studio, Code:Blocks en Xcode. Het is natuurlijk niet mogelijk
om alle mogelijkheden van zulke IDE's te beschrijven (of af te beelden).
Zo'n beetje alle programmavoorbeelden van enige omgang zijn getest op
Visual Studio. Dit is ook de meest gebruikte IDE. De IDE wordt zeker niet
alleen gebruikt voor het ontwikkelen van C-programma's. Bekend is bijvoorbeeld
de taal C\# die veel wordt gebruikt bij het ontwikkelen van programma's op
het Windows-besturingssysteem.

Hoewel C al een oude taal is, wordt het nog volop gebruikt. De reden daarvoor is
dat de C-statements door de compiler worden vertaald naar instructies die de
microprocessor direct kan uitvoeren. Daarom wordt de taal veel gebruikt bij
het schrijven van programma's op microcontrollers. Dit zijn kleine computersystemen
zonder besturingssysteem. Op deze systemen is C (en C++) de enige taal die
gebruikt kan worden. Maar C wordt ook gebruikt op systemen die wel een
besturingssystemen draaien. Zo worden bij het Windows-besturingssysteem
zogenoemde \textsl{drivers} in C of C++ geschreven. Andere talen zijn niet
toepasbaar. Bekend is ook dat het Linux-besturingssysteem grotendeels is
geschreven in C.

Dit boek is opgemaakt in \LaTeX~\cite{latexwebsite} (\LaTeX-engine = \booktexbanner).
\LaTeX\@ leent zich uitstekend
voor het opmaken van lopende tekst, tabellen, figuren, programmacode, vergelijkingen
en referenties. De gebruikte \LaTeX-distributie is TexLive uit 2020~\cite{texlivewebsite}.
Als editor is TexStudio~\cite{texstudiowebsite} gebruikt. Tekst is gezet in
\ifnum 0\ifxetex 1\fi\ifluatex 1\fi>0
Calibri~\cite{calibrifont}, een van de standaard fonts op een bekend besturingssysteem.
Code is opgemaakt in Consolas~\cite{consolasfont}
\else
Charter~\cite{charterfont}, een van de standaard fonts in \LaTeX.
De keuze hiervoor is dat het een prettig te lezen lettertype is, een
\textsl{slanted} letterserie heeft en een bijbehorende wiskundige tekenset heeft.
Code is opgemaakt in Nimbus Mono~\cite{nimbusfont}
\fi
met behulp van de \textsl{listings}-package~\cite{listingsctan}.
Voor het
tekenen van array's, pointers en flowcharts is \textsl{TikZ/PGF} gebruikt~\cite{tikzctan}.
Alle figuren zijn door de auteur zelf ontwikkeld, behalve de logo's van Creative
Commons en De Haagse Hogeschool.

Natuurlijk zullen docenten ook nu opmerken dat dit boek niet aan al hun
verwachtingen voldoet. Dat zal altijd wel zo blijven. Daarom is de broncode
van dit boek vrij beschikbaar. Eenieder die dit wil, kan de inhoud
vrijelijk aanpassen, mits wijzigingen gedeeld
worden. Hiermee wordt dit boek een levend document dat nooit af is.
De broncode
van dit boek is beschikbaar op \url{https://github.com/jesseopdenbrouw/book_c}.



\subsubsection*{Leeswijzer}
Hoofdstuk~\ref{cha:tour} geeft in vogelvlucht enkele belangrijke concepten van  C weer. Er wordt gesproken over wat een computer en een compiler is en hoe een uitvoerbaar bestand wordt gegenereerd. We beginnen met het geven van voorbeelden die de lezer in ontwikkelsoftware kan invoeren om zo de uitvoer te bekijken. We beschrijven kort hoe uitvoer naar beeldscherm en invoer van het toetsenbord wordt gerealiseerd. We behandelen het concept van variabelen en datatypes. Verder wordt even stilgestaan bij beslissen en herhalen en functies. In dit hoofdstuk worden geen pointers, structures en bestandsbewerkingen behandeld.

Hoofdstuk~\ref{cha:vardatexp} gaat dieper in op variabelen, constanten en expressies. Alle beschikbare datatypes worden behandeld alsmede enkele vaste-lengte datatypes. Er wordt uitgelegd wat een expressie is en hoe expressies kunnen worden gebruikt bij berekeningen en beslissingen. C kent een aantal eigenaardige taalconstructies zoals de increment en decrement operatoren, de conditionele expressie en de komma-operator. Een aantal operatoren wordt in dit hoofdstuk niet behandeld zoals de adres- en dereferentie-operatoren, de element-operator en de member-operatoren. Die volgen in de verdere hoofdstukken.

Hoofdstuk~\ref{cha:programmabesturing} laat zien hoe de executie van een programma kan worden beïnvloed met beslissingen en herhalingen. We behandelen het gebruik van beslissingen aan de hand van een bepaalde voorwaarde (ook wel conditie genoemd). Een beslissing kan uitgebreid worden met een programmadeel dat uitgevoerd moet worden als een conditie waar of onwaar is. Herhalingen komen zeer vaak voor in een programma. We behandelen drie voorkomende taalconstructies hiervoor.

Hoofdstuk~\ref{cha:flowcharts} is een hoofdstuk over het documenteren van een (deel van een) programma met behulp van flowcharts en ligt in het verlengde van hoofdstuk~\ref{cha:programmabesturing}. De symbolen worden uitgelegd en
getoond wordt hoe bepaalde taalconstructies op grafische wijze kunnen worden beschreven.

Hoofdstuk~\ref{cha:functies} beschrijft het gebruik van functies om een programma op te delen in kleine eenheden die uitgevoerd kunnen worden. We laten zien hoe een functie gegevens meekrijgt en hoe gegevens worden teruggegeven. Verder laten we zien hoe functies zichzelf kunnen aanroepen: de recursieve functie. We hebben een paragraaf opgenomen over de \textsl{computational complexity} van een recursieve variant van de reeks van Fibonacci. We behandelen verder het gebruik van lokale en globale variabelen en hoe de zichtbaarheid (scope) geregeld is.

Hoofdstuk~\ref{cha:arrays} gaat over array's. We tonen hoe een array wordt gedefinieerd en kan worden gebruikt. We laten zien hoe array's als argument aan functies kan worden meegegeven. We behandelen strings, een array van karakters en laten zien dat voor het bewerken van strings diverse functies beschikbaar zijn.

Hoofdstuk~\ref{cha:pointers} gaat over pointers. Er wordt uitgelegd wat een pointer is en wat de relatie met array's is. Verder wordt beschreven hoe pointers een rol spelen in het doorgeven van veel informatie aan functies. Pointerstructuren kunnen bijzonder complex zijn en we schuwen dan ook niet om pointers-naar-pointers te behandelen. Het gebruik van command line argumenten wordt uitgelegd zodat een uitvoerbaar programma kan worden gestart met optionele argumenten. We besluiten het hoofdstuk maar een inleiding in het gebruik van dynamische geheugenallocatie.

Hoofdstuk~\ref{cha:structures} gaat over structures en unions die handig zijn bij het gebruik van complexe datastructuren. We leggen uit wat een structure is en hoe de gegevens binnen een structure kunnen worden benaderd. Structuren kunnen als argumenten en returnwaarde van functies dienen maar als de structures erg groot zijn is het handiger om pointers te gebruiken. We beschrijven kort de union als een probaat middel om de hoeveelheid gebruikte geheugen in te dammen (alhoewel dat bij moderne PC's eigenlijk geen rol speelt). Ook bitfields worden behandeld, alhoewel bitfields in de praktijk nauwelijks worden gebruikt. Het gebruik van self-referential structures (voor onder andere \textsl{linked lists}) wordt als verdiepend behandeld, hoewel het geen onderdeel van de taal zelf is.

Hoofdstuk~\ref{cha:io} gaat over invoer en uitvoer. Dit zijn geen onderdelen van de taal zelf maar komt zo vaak voor dat er er flink stuk aan wijden. We kijken wat dieper naar het afdrukken op het scherm en het inlezen van het toetsenbord. Natuurlijk behandelen we ook hoe we informatie in bestanden kunnen bewerken: hoe worden bestanden geopend, geschreven, gelezen en gesloten en waar moeten we op letten als we met bestanden werken.

Hoofdstuk~\ref{cha:preprocessor} gaat over de preprocessor. De preprocessor is een faciliteit die vóór de ``echte'' C-compiler wordt gebruikt. We laten zien hoe we header-bestanden kunnen inlezen, hoe we macro's kunnen definiëren en hoe we macro's kunnen gebruiken voor conditionele compilatie.

Hoofdstuk~\ref{cha:compilatieproces} gaat over het compilatieproces. We behandelen hoe het proces in elkaar steekt en wat er allemaal bij komt kijken voordat een uitvoerbaar programma is gerealiseerd. We laten zien hoe we een C-programma over meerdere bestanden kunnen verdelen en hoe we een bibliotheek kunnen realiseren.

In bijlage~\ref{cha:asciitabel} worden de eerste 128 karakters van de ASCII-code behandeld. Diverse karakters zijn niet afdrukbaar, maar worden gebruikt om besturingscommando's tussen systemen te realiseren. In bijlage~\ref{cha:visualstudio} is een korte tutorial beschreven voor het ontwikkelen van C-programma's met Visual Studio. In bijlage~\ref{cha:voorrang} is de complete lijst te zien met voorrangsregels van operatoren in C.

Een aantal paragrafen is gekenmerkt met een \texttt{*} voor het paragraafnummer. Deze paragrafen bevatten verdiepende stof en kunnen bij een eerste cursus worden overgeslagen.

\subsubsection*{Studiewijzer}
Om de taal te leren zouden eigenlijk alle hoofdstukken aan bod moeten komen. We hebben er naar gestreefd om alleen de taal te behandelen en geen onnodige informatie te verschaffen. 

\subsubsection*{Verantwoording inhoud}
Dit boek voldoet alleen aan aandachtspunt 4,01 van de basis Basic of Knowledge and Skills (BoKS) Elektrotechniek. De
BoKS is te vinden via~\cite{hboengineering2016boks}. Eigenlijk vertelt het boek alle kennis die elke beginnende C-programmeur moet weten. Het boek gaat niet in op het gebruik van C in microcontroller-omgevingen, zoals Arduino.

\subsubsection*{Gebruik van Engelse woorden}
Het boek is doorspekt met Engelse woorden. Dat heeft te maken met het gangbare gebruik van deze woorden. Zo spreken we in dit boek over pointers in plaats van het Nederlandse woord wijzers. Aan de andere kan gebruiken we het woord lus bij herhalingen en niet van het Engelse woord loop. Het een en ander komt ook voort uit de persoonlijke voorkeur van de auteur.

\subsubsection*{Website}
Op de website \url{http://ds.opdenbrouw.nl} zijn slides, practicumopdrachten en aanvullende informatie te vinden. De laatste versie van dit boek wordt hierop gepubliceerd. Er zijn ook voorbeeldprojecten voor Visual Studio en Code::Blocks te vinden.

\subsubsection*{Dankbetuigingen}
Dit boek had niet tot stand kunnen komen zonder hulp van een aantal mensen. Ik wil collega Harry Broeders van Hogeschool Rotterdam bedanken voor zijn onuitputtelijke stroom van correcties en opmerkingen. Collega Kees de Joode wordt bedankt voor zijn bijdrage aan de wiskundige functies die in dit boek beschreven zijn. Jon van den Helder en Gerard Tuk worden bedankt voor enkele rake opmerkingen en verbeteringen. René Terhorst wordt bedankt voor het opsporen en verbeteringen van enkele fouten en voor informatie over de verschillende C-standaarden. Verder wil ik Piet op den Brouw (mijn vader, een die-hard C programmeur) bedanken voor het opsporen van fouten en het geven van suggesties.

De ASCII-tabel op pagina~\pageref{tab:talascii-code} is ontwikkeld door Victor Eijkhout. Deze tabel kan gevonden worden op \url{http://www.ctan.org/tex-archive/info/ascii-chart}.


\bigskip
\hfill \author, \ifcase\month \or januari\or februari\or maart\or april\or mei\or juni\or juli\or augustus\or september\or oktober\or november\or december\fi\ \the\year.