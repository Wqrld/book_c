%%%%%%%%%%%%%%%%%%%%%%%%%%%%%%%%%%%%%%%%%%%%%%%%%%%%%%%%%%%%%%%%%%%%%%%%%%%%%%%
%%%
%%%   VOORWOORD
%%%
%%%%%%%%%%%%%%%%%%%%%%%%%%%%%%%%%%%%%%%%%%%%%%%%%%%%%%%%%%%%%%%%%%%%%%%%%%%%%


\chapter{Voorwoord}
\label{cha:voorwoord}
\thispagestyle{empty}

 Een van de beste boeken over C is
\textsl{The C Programming Language} van Brian Kernighan en Dennis
Ritchie~\cite{kernighan1988c}. Het boek is kort en bondig maar laat toch
ruimte om de tekst te ondersteunen met voorbeelden. Helaas zijn veel van
de voorbeelden lastig te volgen voor de beginnende programmeur. Zoals de
schrijvers zelf opmerken:

\begin{displayquote}
The book is not an introductory programming manual; it assumes some familiarity
with basic programming concepts like variables, assignment statements,
loops, and functions.
[...]
C is not a big language, and it is not well served by a big book.
\end{displayquote}

Dit boek is bedoeld als inleiding op de programmeertaal C en is niet
toereikend om alle facetten van de taal te leren. Dat is ook niet de
bedoeling van dit boek. We laten alleen maar zien wat gebruikelijk is bij
het ontwikkelen van C-programma's. 

Veel boeken over C beschrijven het ontwikkelen van programma's op het Unix
operating system en afgeleide varianten zoals Linux, FreeBSD en Mac OS-X.
Dat is echter niet de werkwijze van veel programmeurs. Veel software wordt
ontwikkeld met een Integrated Development Environment (IDE). Bekende IDE's
zijn Visual Studio, Code:Blocks en Xcode. Het is natuurlijk niet mogelijk
om alle mogelijkheden van zulke IDE's te beschrijven (of af te beelden).

Dit boek is opgemaakt in \LaTeX~\cite{latexwebsite} (\LaTeX-engine = \booktexbanner).
\LaTeX\@ leent zich uitstekend
voor het opmaken van lopende tekst, tabellen, figuren, programmacode, vergelijkingen
en referenties. De gebruikte \LaTeX-distributie is TexLive uit 2020~\cite{texlivewebsite}.
Als editor is TexStudio~\cite{texstudiowebsite} gebruikt. Tekst is gezet in
\ifnum 0\ifxetex 1\fi\ifluatex 1\fi>0
Calibri~\cite{calibrifont}, een van de standaard fonts op een bekend besturingssysteem.
Code is opgemaakt in Consolas~\cite{consolasfont}
\else
Charter~\cite{charterfont}, een van de standaard fonts in \LaTeX.
De keuze hiervoor is dat het een prettig te lezen lettertype is, een
\textsl{slanted} letterserie heeft en een bijbehorende wiskundige tekenset heeft.
Code is opgemaakt in Nimbus Mono~\cite{nimbusfont}
\fi
met behulp van de \textsl{listings}-package~\cite{listingsctan}.
Voor het
tekenen van array's, pointers en flowcharts is \textsl{TikZ/PGF} gebruikt~\cite{tikzctan}.
Alle figuren zijn door de auteur zelf ontwikkeld, behalve de logo's van Creative
Commons en De Haagse Hogeschool.

Natuurlijk zullen docenten ook nu opmerken dat dit boek niet aan al hun
verwachtingen voldoet. Dat zal altijd wel zo blijven. Daarom wordt de broncode
van vrij beschikbaar. Eenieder die dit wil, kan de inhoud
vrijelijk aanpassen, mits wijzigingen gedeeld
worden. Hiermee wordt dit boek een levend document dat nooit af is.
De broncode
van dit boek is beschikbaar op \url{https://github.com/jesseopdenbrouw/book_c}.



\subsubsection*{Leeswijzer}
Hoofdstuk~\ref{cha:tour}


\subsubsection*{Studiewijzer}
In onderstaande tabel wordt een overzicht gegeven van de stof die een student
bij een eerste introductie onderwezen zou moeten krijgen. Uiteraard is iedereen
vrij om zelf de onderwerpen te kiezen.

\begin{tabbing}
\hspace{1em}\=\hspace{3cm}\=\kill
   \> \textbf{Tekst} \> \\
\end{tabbing}


\subsubsection*{Verantwoording inhoud}
Dit boek voldoet alleen aan aandachtspunt 4,01 van
de basis Basic of Knowledge and Skills (BoKS) Elektrotechniek. De
BoKS is te vinden via~\cite{hboengineering2016boks}. 

\subsubsection*{Website}
Op de website \url{http://ds.opdenbrouw.nl} zijn slides, practicumopdrachten
en aanvullende informatie te vinden. De laatste versie van dit boek wordt
hierop gepubliceerd. Er zijn ook voorbeeldprojecten voor Visual Studio en
Code::Blocks te vinden.

\subsubsection*{Dankbetuigingen}
Dit boek had niet tot stand kunnen komen zonder hulp van een aantal mensen.
Ik wil collega Harry Broeders van Hogeschool Rotterdam bedanken voor zijn
geweldige bijdrage. Niet alleen op technisch gebied, maar ook taalkundig en
op de indeling van dit boek. Collega Ben Kuiper heeft een waardevolle bijdrage
geleverd op technisch en taalkundig gebied.

De ASCII-tabel op pagina~\pageref{tab:talascii-code} is ontwikkeld door Victor
Eijkhout. Deze tabel kan gevonden worden op
\url{http://www.ctan.org/tex-archive/info/ascii-chart}.


%\bigskip%
\hfill \author, \the\year.
