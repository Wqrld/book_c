\chapter{Flowcharts}
\thispagestyle{empty}
\label{cha:flowcharts}

Een \textsl{flowchart}\index{flowchart} (Nederlands: stroomschema)\index{stroomschema}
is een grafische weergave van een (deel van een) computerprogramma te documenteren.
Flowcharts gebruiken rechthoeken, ovalen, ruiten en mogelijk andere vormen om het programma stap voor stap weer te gegeven, samen met verbindingspijlen die de stroom en volgorde te definiëren. Ze
kunnen variëren van eenvoudige, met de hand getekende diagrammen tot uitgebreide
computergegenereerde diagrammen die meerdere stappen en routes weergeven. Als we alle
verschillende vormen van stroomdiagrammen beschouwen, zijn ze een van de meest
voorkomende diagrammen op aarde, die door zowel technische als niet-technische mensen
gebruikt worden. Ze zijn gerelateerd aan andere
populaire diagrammen, zoals Data Flow Diagrams (DFD's) en Unified Modeling Language
(UML) activiteitendiagrammen.

Het is belangrijk om een flowchart overzichtelijk te houden. Het heeft geen zin
om op een A4'tje bijvoorbeeld 40 symbolen te tekenen. Afhankelijk van de grootte
van het programma moeten we stukken code ``comprimeren''. Zo zouden we de code voor
het inlezen van tien elementen van een array als \'e\'en statement in een flowchart
kunnen tekenen met de tekst ``lees array in''.

De gebruikte symbolen in een flowchart staan hieronder weergegeven:

\medskip
\begin{minipage}[c]{0.3\textwidth}
\centering
\begin{tikzpicture}\node (start) [startstop] {start};\end{tikzpicture}
\end{minipage}\hfill%
\begin{minipage}[c]{0.68\textwidth}
Dit symbool geeft het begin of het einde van de flowchart aan. Naast ``start'' en ``stop'' kunnen ook de termen ``begin'' en ``einde'' gebruikt worden.
\end{minipage}
\bigskip

\begin{minipage}[c]{0.3\textwidth}
\centering
\begin{tikzpicture}\node (io) [io] {invoer\\uitvoer};\end{tikzpicture}
\end{minipage}\hfill%
\begin{minipage}[c]{0.68\textwidth}
Dit symbool geeft invoer of uitvoer aan, bijvoorbeeld het inlezen van een getal of het afdrukken van tekst.\end{minipage}
\bigskip

\begin{minipage}[c]{0.3\textwidth}
\centering
\begin{tikzpicture}\node (proces) [process] {statements};\end{tikzpicture}
\end{minipage}\hfill%
\begin{minipage}[c]{0.68\textwidth}
Dit symbool wordt gebruikt om statements anders dan invoer en uitvoer aan te duiden. Een statement kan bijvoorbeeld een berekening zijn, maar ook een functie-aanroep. Meerdere sequenti\"ele statements mogen samengenomen worden in \'e\'en symbool.
\end{minipage}
\bigskip

\begin{minipage}[c]{0.3\textwidth}
\centering
\begin{tikzpicture}\node (dec) [decision] {expressie};\end{tikzpicture}
\end{minipage}\hfill%
\begin{minipage}[c]{0.68\textwidth}
Dit symbool geeft een beslissing aan. De expressie kan alleen maar ``waar'' of ``niet waar`` opleveren. Een beslissing levert een vertakking op. De vertakkingen kunnen naar links of rechts zijn \'en naar beneden (dus niet links en rechts). Bij de uitgaande vertakkingen worden de woorden ``yes'' en ``no'' geschreven. Naast
``yes'' en ``no'' kunnen ook de termen ``true'' of ``T'' en ``false'' of ``F'' gebruikt worden.
\end{minipage}
\bigskip

\begin{minipage}[c]{0.3\textwidth}
\centering
\begin{tikzpicture}\draw [arrow] (0,0) -- (2,0);\end{tikzpicture}
\end{minipage}\hfill%
\begin{minipage}[c]{0.68\textwidth}
Een pijl geeft een \textsl{flow} aan. De pijl begint bij een van de symbolen en eindigt bij een symbool of een andere pijl. Bij een symbool komt altijd maar \'e\'en pijl aan.
\end{minipage}
\bigskip

\begin{minipage}[c]{0.3\textwidth}
\begin{tikzpicture}\node (con) [connector] {};\end{tikzpicture}
\centering
\end{minipage}\hfill%
\begin{minipage}[c]{0.68\textwidth}
Een connector kan gebruikt worden om een plek aan te geven waar twee pijlpunten samenkomen. Dit symbool wordt niet in dit boek gebruikt.
\end{minipage}

Voorbeelden van invoer en uitvoer zijn: ``lees $i$'' en ``print $k$''. Voorbeelden
van statements zijn (geen invoer en uitvoer): ``$j = j+1$'' en ``$k=2*j$''.
Voorbeelden van expressies zijn: ``$j<10$'' en ``$a>5 \mathrm{\ en\ } a<25$''.


\section{\texttt{if}-statement}
Een \texttt{if}-statement is te modelleren door middel van een beslissing en een statement.

\begin{lstlisting}[caption=\texttt{if}-statement in C]
if ((*\normalfont\textsl{expressie}*)) {
    (*\normalfont\textsl{statements}*)
}
\end{lstlisting}

De flowchart is gegeven in figuur~\ref{fig:flowif}.

\begin{figure}[!ht]
\centering
\begin{tikzpicture}
\node [ghostnode] (start) {};
\node (dec1) [decision, below =of start] {\textsl{expressie}};
\node (pro1) [process, below =of dec1] {\textsl{statements}};
\node [ghostnode] (stop) [below =of pro1] {};
\node [ghostnode] (stop2) [below of= stop] {};

\draw [arrow, dashed] (start) -- (dec1);
\draw [arrow] (dec1) -- node[anchor=east] {yes} (pro1);
\draw [arrow,-] (pro1) -- (stop) coordinate (entrypoint1);
\draw [arrow] (dec1.east) -- node[anchor=south] {no} ++(1.6,0) |- (entrypoint1);
\draw [ghostarrow] (stop) -- (stop2);
\end{tikzpicture}
\caption{Flowchart van een \texttt{if}-statement.}
\label{fig:flowif}
\end{figure}


\section{\texttt{if-else}-statement}

Een \texttt{if-else}-statement van de vorm

\begin{lstlisting}[caption=\texttt{if-else}-statement in C.]
if ((*\normalfont\textsl{expressie}*)) {
    (*\normalfont\textsl{statements\textsubscript{1}}*)
else {
    (*\normalfont\textsl{statements\textsubscript{2}}*)
}
\end{lstlisting}

is in een flowchart te tekenen zoals te zien is in figuur~\ref{fig:flowifelse}.

\begin{figure}[!ht]
\centering
\begin{tikzpicture}
\node [ghostnode] (start) {};
\node (dec1) [decision, below =of start] {\textsl{expressie}};
\node (pro1) [process, below =of dec1] {\textsl{statements\textsubscript{1}}};
\node (pro2) [process, right =of pro1]  {\textsl{statements\textsubscript{2}}};
\node [ghostnode] (stop) [below =of pro1] {};
\node [ghostnode] (stop2) [below of= stop] {};

\draw [ghostarrow] (start) -- (dec1);
\draw [arrow] (dec1) -- node[anchor=east] {yes} (pro1);
\draw [arrow,-] (pro1) -- (stop) coordinate (entrypoint1);
\draw [arrow] (dec1.east) -| node[anchor=south,pos=0.2] {no} (pro2);
\draw [arrow] (pro2.south) |- (entrypoint1);
\draw [ghostarrow] (stop) -- (stop2);
\end{tikzpicture}
\caption{Flowchart van een \texttt{if-else}-statement.}
\label{fig:flowifelse}
\end{figure}


\section{\texttt{while}-lus en \texttt{do-while}-lus}

Een \texttt{while}-lus heeft de gedaante:

\begin{lstlisting}[caption=while-lus in C.]
while ((*\normalfont\textsl{expressie}*)) {
    (*\normalfont\textsl{statements}*)
}
\end{lstlisting}

wordt weergegeven als flowchart in figuur~\ref{fig:flowhile}.
%
\begin{figure}[!ht]
\centering
\begin{tikzpicture}
\node (start) [ghostnode] {};
\node (start1) [ghostnode, below =of start] {};
\node (dec1) [decision, below =of start1] {\textsl{expressie}};
\node (pro1) [process, below =of dec1] {\textsl{statements}};
\node [ghostnode] (stop) [below =of pro1] {};
\node [ghostnode] (stop2) [below of= stop] {};
\node [ghostnode] (stop3) [below of= stop2] {};

\draw [ghostarrow,-] (start) -- (start1) coordinate (entrypoint1);
\draw [arrow] (start1) -- (dec1);
\draw [arrow] (dec1) -- node[anchor=east] {yes} (pro1);
\draw [arrow] (pro1) -- (stop) |- ++(-3,0) |- (entrypoint1);
\draw [arrow,-] (dec1.east) -- ++(1.5,0) node[anchor=south, midway] {no} |- (stop2);
\draw [ghostarrow] (stop2) -- (stop3);
\end{tikzpicture}
\caption{Flowchart van een \texttt{while}-lus.}
\label{fig:flowhile}
\end{figure}
%
De \texttt{do-while}-lus heeft in C de volgende gedaante:

\begin{lstlisting}[caption=Do-while-lus in C.]
do {
    (*\normalfont\textsl{statements}*)
} while ((*\normalfont\textsl{expressie}*)};
\end{lstlisting}

De flowchart is te vinden in figuur~\ref{fig:flodowhile}.
Let erop dat \textsl{statements} minstens \'e\'en keer wordt uitgevoerd, ook al is \textsl{expressie}
niet waar.

\begin{figure}[!ht]
\centering
\begin{tikzpicture}
\node [ghostnode] (start) {};
\node [ghostnode, below =of start] (start1) {};
\node [process, below =of start1] (pro1) {\textsl{statements}};
\node (dec1) [decision, below =of pro1] {\textsl{expressie}};
\node [ghostnode] (stop) [below =of dec1] {};
\node [ghostnode] (stop2) [below of= stop] {};
\node [ghostnode] (stop3) [below of= stop2] {};

\draw [ghostarrow,-] (start) -- (start1);
\draw [arrow] (start1) -- (pro1) coordinate (entrypoint1);
\draw [arrow] (pro1) -- (dec1);
\draw [arrow] (dec1) -- node[anchor=east] {yes} (stop) |- ++(-3,0) |- (start1);
\draw [arrow,-] (dec1.east) -- ++(1.5,0) node[anchor=south, midway] {no} |- (stop2);
\draw [ghostarrow] (stop2) -- (stop3);
\end{tikzpicture}
\caption{Flowchart van een \texttt{do-while}-lus.}
\label{fig:flodowhile}
\end{figure}


\newpage
\section{\texttt{for}-lus}

Een \texttt{for}-lus van de gedaante:
\newpage
\begin{lstlisting}[caption=\texttt{for}-lus in C.]
for ((*\normalfont\textsl{expressie\textsubscript{1}}*); (*\normalfont\textsl{expressie\textsubscript{2}}*); (*\normalfont\textsl{expressie\textsubscript{3}}*)) {
    (*\normalfont\textsl{statements}*)
}
\end{lstlisting}

is \textsl{semantisch identiek} aan (let op de punt-komma's):

\begin{lstlisting}[caption=\texttt{for}-lus herschreven als \texttt{while}-lus in C.]
(*\normalfont\textsl{expressie\textsubscript{1}}*);
while ((*\normalfont\textsl{expressie\textsubscript{2}}*)} {
    (*\normalfont\textsl{statements}*)
    (*\normalfont\textsl{expressie\textsubscript{3}}*);
}
\end{lstlisting}

Let erop dat \textsl{expressie\textsubscript{3}} n\'a \textsl{statements} komt.
De flowchart is te vinden in figuur~\ref{fig:flowfor}.

\begin{figure}[H]
\centering
\begin{tikzpicture}
\node [ghostnode] (start) {};
\node [process, below =of start] (pro1) {\textsl{expressie\textsubscript{1}};};
\node [ghostnode, below =of pro1] (start1) {};
\node (dec1) [decision, below =of start1] {\textsl{expressie\textsubscript{2}}};
\node (pro2) [process, below =of dec1] {\textsl{statements}};
\node (pro3) [process, below =of pro2] {\textsl{expressie\textsubscript{3}};};
\node [ghostnode] (stop) [below =of pro3] {};
\node [ghostnode] (stop2) [below of= stop] {};
\node [ghostnode] (stop3) [below of= stop2] {};

\draw [ghostarrow] (start) -- (pro1);
\draw [arrow] (pro1) -- (dec1);
\draw [arrow] (dec1) -- (pro2) node[anchor=east, midway] {yes};
\draw [arrow] (pro2) -- (pro3);
\draw [arrow] (pro3) -- (stop) |- ++(-3,0) |- (start1);
\draw [arrow,-] (dec1.east) -- ++(1.5,0) node[anchor=south, midway] {no} |- (stop2);
\draw [ghostarrow] (stop2) -- (stop3);
\end{tikzpicture}
\caption{Flowchart van een \texttt{for}-lus.}
\label{fig:flowfor}
\end{figure}

We hebben hier overigens de statements \texttt{break} en \texttt{continue} niet opgenomen. Deze statements moeten apart worden ingepast.

\section{\texttt{switch}-statement}
Een \texttt{switch}-statement is een meervoudige \texttt{if}-statement met steeds dezelfde variabele.

\begin{lstlisting}[caption=switch-statement in C.]
switch ((*\normalfont\textsl{variabele}*)) {
    case (*\normalfont\textsl{expressie\textsubscript{1}}*): (*\normalfont\textsl{statements\textsubscript{1}}*); break;
    case (*\normalfont\textsl{expressie\textsubscript{2}}*): (*\normalfont\textsl{statements\textsubscript{2}}*); break;
    ...
    case (*\normalfont\textsl{expressie\textsubscript{n}}*): (*\normalfont\textsl{statements\textsubscript{n}}*); break;
    default : (*\normalfont\textsl{statements\textsubscript{d}}*); break;
}
\end{lstlisting}

Merk op dat \textsl{expressie\textsubscript{1}}, \textsl{expressie\textsubscript{2}}, \ldots, \textsl{expressie\textsubscript{n}}
een constante moeten opleveren. De flowchart is te zien in figuur~\ref{fig:switch}.

\begin{figure}[!ht]
\centering
\begin{tikzpicture}
\node [ghostnode] (start) {};
\node (dec1) [decision, below =of start] {\textsl{variabele = expressie\textsubscript{1}}};
\node (dec2) [decision, below =of dec1] {\textsl{variabele = expressie\textsubscript{2}}};
\node (decn) [decision, below =of dec2] {\textsl{variabele = expressie\textsubscript{n}}};
\node (prod) [process, below =of decn] {\textsl{statements\textsubscript{d}}};
\node [ghostnode] (stop) [below =of prod] {};
\node [ghostnode] (stop2) [below of= stop] {};
\node (pro1) [process, right =of dec1] {\textsl{statements\textsubscript{1}}};
\node (pro2) [process, right =of dec2] {\textsl{statements\textsubscript{2}}};
\node (pron) [process, right =of decn] {\textsl{statements\textsubscript{n}}};
\node (entrypoint1) [ghostnode, right =of pro1] {};
\node (entrypoint2) [ghostnode, right =of pro2] {};
\node (entrypointn) [ghostnode, right =of pron] {};

\draw [ghostarrow] (start) -- (dec1);
\draw [arrow] (dec1) -- node[anchor=east] {no} (dec2);
\draw [ghostarrow] (dec2) -- node[anchor=east] {no} (decn);
\draw [arrow] (decn) -- node[anchor=east] {no} (prod);
\draw [arrow,-] (prod) -- (stop);
\draw [ghostarrow] (stop) -- (stop2);
\draw [arrow] (dec1) -- node[anchor=south] {yes} (pro1);
\draw [arrow] (dec2) -- node[anchor=south] {yes} (pro2);
\draw [arrow] (decn) -- node[anchor=south] {yes} (pron);
\draw [arrow] (pro1) -- (entrypoint1) |- (stop);
\draw [arrow] (pro2) -- (entrypoint2);
\draw [arrow] (pron) -- (entrypointn);

%%%\draw [arrow,-] (pro1) -- (stop) coordinate (entrypoint1);
%%%\draw [arrow] (dec1.east) -- node[anchor=south] {no} ++(1.6,0) |- (entrypoint1);
%%%\draw [ghostarrow] (stop) -- (stop2);
\end{tikzpicture}
\caption{Flowchart van een \texttt{switch}-statement met \texttt{break}.}
\label{fig:switch}
\end{figure}

Merk op dat we hier expliciet het \texttt{break}-statement hebben toegevoegd. Zonder dit statement wordt verder gegaan met het testen van de volgende expressie. Zie listing~\ref{cod:floswitchnobreak}.

\begin{lstlisting}[caption=switch-statement in C.,label=cod:floswitchnobreak]
switch ((*\normalfont\textsl{variabele}*)) {
    ...
    case (*\normalfont\textsl{expressie\textsubscript{1}}*): (*\normalfont\textsl{statements\textsubscript{1}}*);
    case (*\normalfont\textsl{expressie\textsubscript{2}}*): (*\normalfont\textsl{statements\textsubscript{2}}*);
    ...
}
\end{lstlisting}

De flowchart is te zien in figuur~\ref{fig:switch2}.

\begin{figure}[!ht]
\centering
\begin{tikzpicture}
\node [ghostnode] (start) {};
\node (dec1) [decision, below =of start] {\textsl{variabele = expressie\textsubscript{1}}};
\node [ghostnode, below =of dec1] (ff) {};
\node (dec2) [decision, below =of ff] {\textsl{variabele = expressie\textsubscript{2}}};
\node (pro1) [process, right =of dec1] {\textsl{statements\textsubscript{1}}};
\node (pro2) [ghostnode, right =of dec2] {};
\node (entrypoint1) [ghostnode, right =of pro1] {};
\node [ghostnode, below = of dec2] (end) {};

\draw [ghostarrow] (start) -- (dec1);
\draw [ghostarrow] (dec2) -- (pro2);
\draw [ghostarrow] (dec2) -- (end);
\draw [arrow] (dec1) -- node[anchor=east] {no} (dec2);
\draw [arrow] (dec1) -- node[anchor=south] {yes} (pro1);
\draw [arrow] (pro1) -- (entrypoint1) |- (ff);
\end{tikzpicture}
\caption{Flowchart van een \texttt{switch}-statement zonder \texttt{break}.}
\label{fig:switch2}
\end{figure}


\section{Voorbeeld}
We zullen een flowchart presenteren van een kort programma.
Hieronder is de flowchart te zien van een eenvoudig C-programma gegeven dat een variabele $k$ inleest en
vervolgens de som bepaalt van alle getallen tussen 1 en $k$ (inclusief), dus:
%
\begin{equation}
j = 1 + 2 +3 + 4 +\cdots + k
\end{equation}

%\begin{figure}[!ht]
%\begin{lstlisting}[caption=Codevoorbeeld in C.]
%#include <stdio.h>
%
%int main(void) {
%
%    int k, i, j = 0;
%    
%    scanf("%d", &k);
%
%    for (i=1; i<=k; i=i+1) {
%        j=j+i;
%    }
%    
%    printf("%d\n", j);
%        
%    return 0;
%}
%\end{lstlisting}
%\end{figure}

We hebben hier te maken met het begin en einde van het programma, het inlezen en
afdrukken van variabelen en het doorlopen van een for-lus waarbij variabelen worden
aangepast. Uiteindelijk komen we uit bij het einde van het programma.
In figuur~\ref{fig:exampleflowchart} is de flowchart te zien.

\begin{figure}[!ht]
\centering
\begin{tikzpicture}%[node distance=1cm]
\node (start) [startstop] {Start};
\node (in1) [io, below =of start] {Input $k$};
\node (pro1) [process, below =of in1] {$j=0$\\$i=1$};
\node (dec1) [decision, below =of pro1] {$i\leq k$};
\node (pro2a) [process, below =of dec1] {$j=j+i$\\$i=i+1$};
\node (gnode) [ghostnode, below =of pro2a] {};
\node (gstnde) [ghostnode, below =of gnode] {};
\node (out1) [io, below =of gstnde] {Print $j$};
\node (stop) [startstop, below =of out1] {Stop};

\draw [arrow] (start) -- (in1);
\draw [arrow] (in1) -- (pro1);
\draw [arrow] (pro1) -- (dec1) coordinate[midway] (entrypoint1);
\draw [arrow] (dec1) -- node[anchor=east] {yes} (pro2a);
\draw [arrow] (dec1) -- node[anchor=south] {no} ++(3,0) |- (gstnde) -| (out1);
\draw [arrow] (pro2a) -- (gnode) |- ++(-3,0) |- (entrypoint1);
%\draw [arrow] (pro2a) -- (out1);
\draw [arrow] (out1) -- (stop);

%%%\begin{pgfonlayer}{background}
%%%\node[densely dashed, draw=black, fill=yellow!50, fit=(pro1) (dec1) (pro2b) (test),inner sep=10pt] {};
%%%\end{pgfonlayer}
\end{tikzpicture}
\caption{Voorbeeld van een flowchart.}
\label{fig:exampleflowchart}
\end{figure}
